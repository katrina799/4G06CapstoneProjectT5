% THIS DOCUMENT IS FOLLOWS THE VOLERE TEMPLATE BY Suzanne Robertson and James Robertson
% ONLY THE SECTION HEADINGS ARE PROVIDED
%
% Initial draft from https://github.com/Dieblich/volere
%
% Risks are removed because they are covered by the Hazard Analysis
\documentclass[12pt]{article}

\usepackage{booktabs}
\usepackage{tabularx}
\usepackage{hyperref}
\hypersetup{
    bookmarks=true,         % show bookmarks bar?
      colorlinks=true,      % false: boxed links; true: colored links
    linkcolor=red,          % color of internal links (change box color with linkbordercolor)
    citecolor=green,        % color of links to bibliography
    filecolor=magenta,      % color of file links
    urlcolor=cyan           % color of external links
}

\newcommand{\lips}{\textit{Insert your content here.}}

%% Comments

\usepackage{color}

\newif\ifcomments\commentstrue %displays comments
%\newif\ifcomments\commentsfalse %so that comments do not display

\ifcomments
\newcommand{\authornote}[3]{\textcolor{#1}{[#3 ---#2]}}
\newcommand{\todo}[1]{\textcolor{red}{[TODO: #1]}}
\else
\newcommand{\authornote}[3]{}
\newcommand{\todo}[1]{}
\fi

\newcommand{\wss}[1]{\authornote{blue}{SS}{#1}} 
\newcommand{\plt}[1]{\authornote{magenta}{TPLT}{#1}} %For explanation of the template
\newcommand{\an}[1]{\authornote{cyan}{Author}{#1}}

%% Common Parts

\newcommand{\progname}{Course Buddy} % PUT YOUR PROGRAM NAME HERE
\newcommand{\authname}{Team \#5, Overwatch League
\\ Jingyao, Qin
\\ Qianni, Wang
\\ Qiang, Gao
\\ Chenwei, Song
\\ Shuting, Shi
\\ } % AUTHOR NAMES                  

\usepackage{hyperref}
    \hypersetup{colorlinks=true, linkcolor=blue, citecolor=blue, filecolor=blue,
                urlcolor=blue, unicode=false}
    \urlstyle{same}
                                

\newcounter{udreqnum} %User documentation requirement number
\newcommand{\rtheudreqnum}{UDR\refstepcounter{udreqnum}\theudreqnum:}
\newcommand{\udrrref}[1]{UDR\ref{#1}}
\newcounter{utreqnum} %User training requirement number
\newcommand{\rtheutreqnum}{UTR\refstepcounter{utreqnum}\theutreqnum:}
\newcommand{\utrref}[1]{UTR\ref{#1}}

\begin{document}

\title{Software Requirements Specification for \progname: subtitle describing software} 
\author{\authname}
\date{\today}
	
\maketitle

~\newpage

\pagenumbering{roman}

\tableofcontents

~\newpage

\section*{Revision History}

\begin{tabularx}{\textwidth}{p{3cm}p{2cm}X}
\toprule {\textbf{Date}} & {\textbf{Version}} & {\textbf{Notes}}\\
\midrule
Date 1 & 1.0 & Notes\\
Date 2 & 1.1 & Notes\\
\bottomrule
\end{tabularx}

~\\

~\newpage
\section{Purpose of the Project}
\subsection{User Business}
\lips
\subsection{Goals of the Project}
\lips
\section{Stakeholders}
\subsection{Client}
\lips
\subsection{Customer}
\lips
\subsection{Other Stakeholders}
\lips
\subsection{Hands-On Users of the Project}
\lips
\subsection{Personas}
\lips
\subsection{Priorities Assigned to Users}
\lips
\subsection{User Participation}
\lips
\subsection{Maintenance Users and Service Technicians}
\lips

\section{Mandated Constraints}
\subsection{Solution Constraints}
\lips
\subsection{Implementation Environment of the Current System}
\lips
\subsection{Partner or Collaborative Applications}
\lips
\subsection{Off-the-Shelf Software}
\lips
\subsection{Anticipated Workplace Environment}
\lips
\subsection{Schedule Constraints}
\lips
\subsection{Budget Constraints}
\lips
\subsection{Enterprise Constraints}
\lips

\section{Naming Conventions and Terminology}
\subsection{Glossary of All Terms, Including Acronyms, Used by Stakeholders
involved in the Project}
\lips

\section{Relevant Facts And Assumptions}
\subsection{Relevant Facts}
\lips
\subsection{Business Rules}
\lips
\subsection{Assumptions}
\lips

\section{The Scope of the Work}
\subsection{The Current Situation}
\lips
\subsection{The Context of the Work}
\lips
\subsection{Work Partitioning}
\lips
\subsection{Specifying a Business Use Case (BUC)}
\lips

\section{Business Data Model and Data Dictionary}
\subsection{Business Data Model}
\lips
\subsection{Data Dictionary}
\lips

\section{The Scope of the Product}
\subsection{Product Boundary}
\lips
\subsection{Product Use Case Table}
\lips
\subsection{Individual Product Use Cases (PUC's)}
\lips

\section{Functional Requirements}
\subsection{Functional Requirements}
\lips

\section{Look and Feel Requirements}
\subsection{Appearance Requirements}
\lips
\subsection{Style Requirements}
\lips

\section{Usability and Humanity Requirements}
\subsection{Ease of Use Requirements}
\lips
\subsection{Personalization and Internationalization Requirements}
\lips
\subsection{Learning Requirements}
\lips
\subsection{Understandability and Politeness Requirements}
\lips
\subsection{Accessibility Requirements}
\lips

\section{Performance Requirements}
\subsection{Speed and Latency Requirements}
\lips
\subsection{Safety-Critical Requirements}
\lips
\subsection{Precision or Accuracy Requirements}
\lips
\subsection{Robustness or Fault-Tolerance Requirements}
\lips
\subsection{Capacity Requirements}
\lips
\subsection{Scalability or Extensibility Requirements}
\lips
\subsection{Longevity Requirements}
\lips

\section{Operational and Environmental Requirements}
\subsection{Expected Physical Environment}
\lips
\subsection{Wider Environment Requirements}
\lips
\subsection{Requirements for Interfacing with Adjacent Systems}
\lips
\subsection{Productization Requirements}
\lips
\subsection{Release Requirements}
\lips

\section{Maintainability and Support Requirements}
\subsection{Maintenance Requirements}
\lips
\subsection{Supportability Requirements}
\lips
\subsection{Adaptability Requirements}
\lips

\section{Security Requirements}
\subsection{Access Requirements}
\lips
\subsection{Integrity Requirements}
\lips
\subsection{Privacy Requirements}
\lips
\subsection{Audit Requirements}
\lips
\subsection{Immunity Requirements}
\lips

\section{Cultural Requirements}
\subsection{Cultural Requirements}
\lips

\section{Compliance Requirements}
\subsection{Legal Requirements}
\lips
\subsection{Standards Compliance Requirements}
\lips

\section{Open Issues}
\lips

\section{Off-the-Shelf Solutions}
\subsection{Ready-Made Products}
\lips
\subsection{Reusable Components}
\lips
\subsection{Products That Can Be Copied}
\lips

\section{New Problems}
\subsection{Effects on the Current Environment}
\lips
\subsection{Effects on the Installed Systems}
\lips
\subsection{Potential User Problems}
\lips
\subsection{Limitations in the Anticipated Implementation Environment That May
Inhibit the New Product}
\lips
\subsection{Follow-Up Problems}
\lips

\section{Tasks}
\subsection{Project Planning}
\lips
\subsection{Planning of the Development Phases}
\lips

\section{Migration to the New Product}
\subsection{Requirements for Migration to the New Product}
\lips
\subsection{Data That Has to be Modified or Translated for the New System}
\lips

\section{Costs}
\lips
\section{User Documentation and Training}
\subsection{User Documentation Requirements}

\begin{itemize}[leftmargin=16.5mm,labelsep=4mm,label=\rtheudreqnum]
\item A user manual that covers all features of the website should be provided in a digital format, written in clear language complete with diagrams. \\
\item User manual should be updated with each release to reflect any changes made. \\
\item A help/support channel should be provided so that users can ask questions and give feedback which will be taken into consideration for future developments and improvements.\\
\end{itemize}

\subsection{Training Requirements}
\begin{itemize}[leftmargin=16.5mm,labelsep=4mm,label=\rtheutreqnum]
\item A short tutorial should be provided if it is the user’s first time using the website. \\
\item A training video should be provided with demos on how to use major features of the website. \\
\item Training videos should be added or updated with each release to reflect any changes.\\
\end{itemize}

\section{Waiting Room}
\begin{itemize}
  \item Add a dark mode support for the website
  \item Add a colour blind support for the website
  \item Add sound effects to the website to make it more user friendly and interesting
  \item Implement a feature where the users are able to integrate with other popular music apps like Spotify to play their favourite playlists while studying
  \item Implement a badges or achievements system such as points for completing tasks which can make the use of the website more engaging and rewarding 
  \item Provide data analysis and insight on how the user performs within a period of time and provide feedback on how to improve their way of study
  \item Add support to other platforms such as mobile applications and desktop applications
\end{itemize}

\section{Ideas for Solution}
\subsection{Idea 1}
\textbf{Requirement}: \frref{R_upload_pdf}, The user could upload multiple PDF files containing course outlines to the system. \\
\textbf{Potential solution}: We could use a Python web framework Flask to set up a web app which supports accessing and uploading multiple files with PDF extensions.\\
\textbf{Advantages}: This will allow users to upload their PDF easily and more securely, and with no additional effort required for developers to implement and test these features, as Flask has already taken care of it for us.\\
\textbf{Disadvantages}: Using Flask to support this feature doesn't present any foreseeable disadvantages in the near future. \\

\subsection{Idea 2}
\textbf{Requirement}: \frref{R_prioritize_task}, The system could assign priority to each task generated. \\
\textbf{Potential solution}: A inference pipeline could be set up to give inference on the unclassified tasks using taskType, weight and deadline as the inputs. The inference pipeline could use a trained model of task priority classification.\\
\textbf{Advantages}: Setting an inference pipeline could be more flexible and could potentially give better results than implementing an fixed algorithm for giving predictions. Several steps for the inference pipeline could be constructed and it will be easier to make changes once it gets set up, simply adding, changing or removing steps when there is a future requirement.\\
\textbf{Disadvantages}: There will be an overload of defining each step and connecting them. \\

\subsection{Idea 3}
\textbf{Requirement}: FR\ref{R_check_task_progress}, The user should be able to view the progress of each task. \\
\textbf{Potential solution}: Could implement a cloud based solution for holding and retrieving data of task, time, etc. A potential choice could be using AWS S3\\
\textbf{Advantages}: This would allow users to check their progress of task and time in a more available, reliable, and well-managed service. The data can be well secured and can be accessed in many ways for example using AWS S3, AWS console, AWS Command Line, providing multiple ways for developers to implement  \\
\textbf{Disadvantages}: There is a cost associated with it when using third party cloud services
\newpage{}
\section*{Appendix --- Reflection}

The information in this section will be used to evaluate the team members on the
graduate attribute of Lifelong Learning.  Please answer the following questions:

\begin{enumerate}
  \item What knowledge and skills will the team collectively need to acquire to
  successfully complete this capstone project?  Examples of possible knowledge
  to acquire include domain specific knowledge from the domain of your
  application, or software engineering knowledge, mechatronics knowledge or
  computer science knowledge.  Skills may be related to technology, or writing,
  or presentation, or team management, etc.  You should look to identify at
  least one item for each team member. \\
  
  The following knowledge and skills will the team need to acquire to successfully complete this capstone project:
  \begin{itemize}
    \item GitHub features such as issue tracker, using workflows/Github Actions, setting up Dev Containers, etc.
    \item Machine learning and neural network knowledge for the developing and optimizing task priority classification models
    \item Machine learning libraries and frameworks such as pandas, scikit-learn, and Keras
    \item Python web framework for web development such Streamlit and Flask
    \item Agile methodologies such as sprint planning and for team management, efficient collaborations and constant delivery, which is important for incremental and iterative development like this project
  \end{itemize}
  \item For each of the knowledge areas and skills identified in the previous
  question, what are at least two approaches to acquiring the knowledge or
  mastering the skill?  Of the identified approaches, which will each team
  member pursue, and why did they make this choice?
  \begin{table}[]
    \begin{tabular}{| p{3cm} | p{3.5cm} | p{2cm} | p{5cm} |}
    \hline
      \textbf{Knowledge or Skills} & \textbf{Approaches} & \textbf{Assigned Team Member} & \textbf{Reason} \\
    \hline
      GitHub & \raggedright Use ChatGPT, Google, watch online tutorials, or ask supervisor for help  & Shuting, Shi & Strong interest in using GitHub for project management, previous experience with other projects. \\
    \hline
      \raggedright Machine learning and neural network & \raggedright Use ChatGPT, Google, watch online tutorials, and read research papers & Qiang, Gao & Strong interest in ML and neural network, watched many online tutorials and read many related books. \\
    \hline
      \raggedright Machine learning libraries and framework & \raggedright Use ChatGPT, Google, watch online tutorials, or ask supervisor for help & Qianni, Wang & Experience with many ML projects where these libraries are being used in AI programs and previous co-op work terms. \\
    \hline
      \raggedright Python web framework & \raggedright Use ChatGPT, Google, watch online tutorials, or ask supervisor for help & Chenwei, Song & Experience in web development in previous co-op work terms. \\
    \hline
      \raggedright Agile methodologies & \raggedright Use ChatGPT, Google, and watch online tutorials & Jingyao, Qin & Experience with Agile mindset in previous co-op work terms, strong interest in project management as the team lead. \\
    \hline
    \end{tabular}
\end{table}
\end{enumerate}

\end{document}