% THIS DOCUMENT IS FOLLOWS THE VOLERE TEMPLATE BY Suzanne Robertson and James Robertson
% ONLY THE SECTION HEADINGS ARE PROVIDED
%
% Initial draft from https://github.com/Dieblich/volere
%
% Risks are removed because they are covered by the Hazard Analysis
\documentclass[12pt]{article}

\usepackage{booktabs}
\usepackage{tabularx}
\usepackage{hyperref}
\usepackage{enumitem}
\usepackage{longtable}
\hypersetup{
    bookmarks=true,         % show bookmarks bar?
      colorlinks=true,      % false: boxed links; true: colored links
    linkcolor=red,          % color of internal links (change box color with linkbordercolor)
    citecolor=green,        % color of links to bibliography
    filecolor=magenta,      % color of file links
    urlcolor=cyan           % color of external links
}

\newcommand{\lips}{\textit{Insert your content here.}}
\newcounter{reqnum} %Functional requirement number
\newcommand{\rthereqnum}{FR\refstepcounter{reqnum}\thereqnum:}
\newcommand{\frref}[1]{FR\ref{#1}}

\newcounter{uhrnum} %Usability and Humanity Requirement number
\newcommand{\rtheuhrnum}{UHR\refstepcounter{uhrnum}\theuhrnum:}
\newcommand{\uhrref}[1]{UHR\ref{#1}}

\newcounter{oernum} %Operational and Environmental Requirement number
\newcommand{\rtheoernum}{OER\refstepcounter{oernum}\theoernum:}
\newcommand{\oerref}[1]{OER\ref{#1}}

\newcounter{srnum} %Security Requirement number
\newcommand{\rthesrnum}{SR\refstepcounter{srnum}\thesrnum:}
\newcommand{\srref}[1]{SR\ref{#1}}

\newcounter{crnum} %Compliance  Requirement number
\newcommand{\rthecrnum}{CR\refstepcounter{crnum}\thecrnum:}
\newcommand{\crref}[1]{CR\ref{#1}}

%% Comments

\usepackage{color}

\newif\ifcomments\commentstrue %displays comments
%\newif\ifcomments\commentsfalse %so that comments do not display

\ifcomments
\newcommand{\authornote}[3]{\textcolor{#1}{[#3 ---#2]}}
\newcommand{\todo}[1]{\textcolor{red}{[TODO: #1]}}
\else
\newcommand{\authornote}[3]{}
\newcommand{\todo}[1]{}
\fi

\newcommand{\wss}[1]{\authornote{blue}{SS}{#1}} 
\newcommand{\plt}[1]{\authornote{magenta}{TPLT}{#1}} %For explanation of the template
\newcommand{\an}[1]{\authornote{cyan}{Author}{#1}}

%% Common Parts

\newcommand{\progname}{Course Buddy} % PUT YOUR PROGRAM NAME HERE
\newcommand{\authname}{Team \#5, Overwatch League
\\ Jingyao, Qin
\\ Qianni, Wang
\\ Qiang, Gao
\\ Chenwei, Song
\\ Shuting, Shi
\\ } % AUTHOR NAMES                  

\usepackage{hyperref}
    \hypersetup{colorlinks=true, linkcolor=blue, citecolor=blue, filecolor=blue,
                urlcolor=blue, unicode=false}
    \urlstyle{same}
                                

\begin{document}

\title{Software Requirements Specification for \progname: subtitle describing software} 
\author{\authname}
\date{\today}
	
\maketitle

~\newpage

\pagenumbering{roman}

\tableofcontents

~\newpage

\section*{Revision History}

\begin{tabularx}{\textwidth}{p{3cm}p{2cm}X}
\toprule {\textbf{Date}} & {\textbf{Version}} & {\textbf{Notes}}\\
\midrule
Date 1 & 1.0 & Notes\\
Date 2 & 1.1 & Notes\\
\bottomrule
\end{tabularx}

~\\

~\newpage
\section{Purpose of the Project}
\subsection{User Business}
The users would be students in high school to college institutes all over the world dealing with multiple courses.

\subsection{Goals of the Project}
We aim to develop a tool to help students integrate multiple course information with their personal schedules to generate a customized study plan. The plan should dynamically self-adjust according to the user's preference and study efficiency so that students can finish their deliverables before the deadline at their own pace with minimal pressure.
\section{Stakeholders}
\subsection{Client}
\lips
\subsection{Customer}
\lips
\subsection{Other Stakeholders}
\lips
\subsection{Hands-On Users of the Project}
\lips
\subsection{Personas}
\lips
\subsection{Priorities Assigned to Users}
\lips
\subsection{User Participation}
\lips
\subsection{Maintenance Users and Service Technicians}
\lips

\section{Mandated Constraints}
\subsection{Solution Constraints}
The development of a fully functional product is required to be finished by February 5 when the Revision 0 Demonstration is scheduled.
\subsection{Implementation Environment of the Current System}
The project implementation would be done through \textit{VS Code} in the beginning and converted into \textit{Github Codespace} once set up gets finished to ensure consistent compiling performance among group members.
\subsection{Partner or Collaborative Applications}
Our system would import event data from and export generated study plans to popular calendar applications like \textit{Google Calendar} and \textit{Outlook Calendar}. 
\subsection{Off-the-Shelf Software}
\subsubsection{StudySchedule.org}
StudySchedule is a free scheduling software dedicated to generating customized daily schedules for students to study for MCAT. They would ask the user to set up an account, pick study material from their MCAT resource library, and take a questionnaire on time constraints and pace preference. Students could view their progress and make adjustments as they wish.
\subsubsection{Taskade AI Genrator}
Taskade is a powerful team project management tool. The component \textit{Taskade AI} is capable of generating tasks for given project topics and creating checklists, project plans, and calendar schedules. \textit{Taskade AI} is also capable of summarising \textit{PDF} files.
\subsection{Anticipated Workplace Environment}
Our web-based app is anticipated to be compatible with mainstream browsers including \textit{Chrome, Safari, Microsoft Edge}, and \textit{Firefox} on the latest version of \textit{Windows, Linux} and \textit{macOS}.
\subsection{Schedule Constraints}
Each member of our team would devote 8 hours per week to work on the project making a total of 40 hours per week.
\subsection{Budget Constraints}
The monetary expenditure for the entire project could not exceed \$750.
\subsection{Enterprise Constraints}
N/A

\section{Naming Conventions and Terminology}
\subsection{Glossary of All Terms, Including Acronyms, Used by Stakeholders
involved in the Project}
\subsection{Table of Units}

Throughout this document, SI (Syst\`{e}me International d'Unit\'{e}s) is employed
as the unit system.  In addition to the basic units, several derived units are
used as described below.  For each unit, the symbol is given followed by a
description of the unit and the SI name.

\begin{tabular}{ |l|l|l|  }

\hline
symbol & unit & SI \\
\hline
\texttt{s} & time & second\\
\hline
\end{tabular}

\subsection{Symbolic Parameters}

\begin{tabular}{|l|l|l|p{5cm}|}

\hline
parameter & value & unit & description\\
\hline
\texttt{MIN\_EDUCATION} & high school & N/A & the minimum level of education\\
\hline
\texttt{MIN\_UNDERSTAND\%} & 95 & N/A & the minimum percentage of testers who can understand among all testers\\
\hline
\texttt{MAX\_TRIAL\_TIME} & 1200 & \texttt{s} & the maximum allowed trial time\\
\hline
\texttt{MIN\_TESTER\_NUM} &  20& N/A & the minimum number of testers needed\\
\hline
\texttt{MAX\_BAD\_GRAMMAR} & 0& N/A & the maximum occurrence of grammar mistakes allowed \\
\hline
\texttt{MAX\_OFFENSIVE} & 0& N/A & the maximum occurrence of offensive messages allowed\\
\hline
\texttt{MAX\_COLOR\_AMBIGUOUS} & 0& N/A & the maximum occurrence of indistinguishable color combinations allowed\\
\hline
\texttt{MIN\_OPERABLE\%} & 95 & N/A & the minimum percentage of system being operable \\
\hline
\texttt{MIN\_API\_SUCCESS\%} & 95 & N/A &  the minimum percentage of successful API calls\\
\hline
\texttt{MIN\_REGRESSION\_PASS\%} & 100  & N/A &  the minimum percentage of successful API calls\\
\hline
\end{tabular}


\section{Relevant Facts And Assumptions}
\subsection{Relevant Facts}
Manually reading through multiple course outlines, inputting all the deliverable information into the calendar, and crunch time has always been an inefficient part of academic life. Students wish to have a seamless tool integrating deadline management into their everyday lives without risking overdue penalties.
\subsection{Business Rules}
N/A
\subsection{Assumptions}
\begin{itemize}
    \item Course outlines are available as \textit{PDF} files.
    \item Users are using \textit{Windows, Linux} and \textit{macOS} operating systems.
    \item Users have access to browsers including \textit{Chrome, Safari, Microsoft Edge}, and \textit{Firefox}.
    \item Users have access to the internet.
    \item Users have basic technical skills including typing, downloading, and uploading files.
    \item Users are using a mainstream calendar application: \textit{Google Calendar, Outlook Calendar, Calendar}.
    \item Users have a relatively stable weekly schedule.
\end{itemize}


\section{The Scope of the Work}
\subsection{The Current Situation}
\lips
\subsection{The Context of the Work}
\lips
\subsection{Work Partitioning}
\lips
\subsection{Specifying a Business Use Case (BUC)}
\lips

\section{Business Data Model and Data Dictionary}
\subsection{Business Data Model}
\lips
\subsection{Data Dictionary}
\subsubsection{Users}

\begin{longtable}{ |p{3cm}|p{5cm}|p{5cm}|}
\hline
\multicolumn{3}{|c|}{Users} \\
\hline
Attribute & userName & userPwd \\
\hline
Description & Unique identifier of a user & The password of an account\\
\hline
Type & \texttt{VARCHAR(50)}& \texttt{VARCHAR(50)}\\
\hline
Allowed Values &N/A &N/A\\
\hline
Default Value &N/A &N/A\\
\hline
Constraints & \texttt{PRIMARY KEY, NOT NULL} & \texttt{NOT NULL CHECK (CHAR\_LENGTH(userPwd) \textgreater 8)}\\
\hline
Source & User input when setting up the account & User input when setting up the account \\
\hline
Usage & Authentication & Authentication \\
\hline
\end{longtable}
\subsubsection{Courses}
\begin{longtable}{ |p{3cm}|p{5cm}|p{5cm}|  }
\hline
\multicolumn{3}{|c|}{Courses} \\
\hline
Attribute & subject & courseCode\\
\hline
Description & The subject of a course& The code of a course\\
\hline
Type & \texttt{VARCHAR(16)}& \texttt{VARCHAR(16)}\\
\hline
Allowed Values &N/A &N/A\\
\hline
Default Value &N/A &N/A\\
\hline
Constraints & \texttt{NOT NULL, PART OF PRIMARY KEY}& \texttt{NOT NULL, PART OF PRIMARY KEY}\\
\hline
Source & Extracted from the course outline & Extracted from the course outline\\
\hline
Usage & Record course information & Record course information\\
\hline
Attribute & courseId&\\
\hline
Description & The unique identifier of a course&\\
\hline
Type & \texttt{VARCHAR(16)}&\\
\hline
Allowed Values & N/A&\\
\hline
Default Value &N/A &\\
\hline
Constraints & \texttt{PRIMARY KEY}&\\
\hline
Source & Uniquely generated when a course outline is uploaded&\\
\hline
Usage & Record course information &\\
\hline
\end{longtable}

\subsubsection{Tasks}
\begin{longtable}[H]{ |p{3cm}|p{5cm}|p{5cm}|  }
\hline
\multicolumn{3}{|c|}{Tasks} \\
\hline
Attribute & weight & taskType\\
\hline
Description & The percentage weight associated with a task & The type of a task\\
\hline
Type & \texttt{DECIMAL(10,2)}& \texttt{INT}\\
\hline
Allowed Values &N/A & \parbox{5cm}{\texttt{0 - QUIZ}\\ \texttt{1 - ASSIGNMENT}\\ \texttt{2 - PRESENTATION}\\ \texttt{3 - MIDTERM}\\ \texttt{4 - EXAM}\\ \texttt{5 - REPORT}\\ \texttt{6 - OTHER}} \\
\hline
Default Value & 0& 0\\
\hline
Constraints & \texttt{NOT NULL CHECK (weight >= 0 AND weight <= 100)}& \texttt{NOT NULL}\\
\hline
Source & Extracted from the course outline & Extracted from the course outline \\
\hline
Usage & Record course information  & Record course information \\
\hline
Attribute & subject & courseCode\\
\hline
Description & The subject of a course& The code of a course\\
\hline
Type & \texttt{VARCHAR(16)}& \texttt{VARCHAR(16)}\\
\hline
Allowed Values &N/A &N/A\\
\hline
Default Value &N/A &N/A\\
\hline
Constraints & \texttt{NOT NULL}& \texttt{NOT NULL}\\
\hline
Source & Extracted from course outline & Extracted from the course outline\\
\hline
Usage & Record course information & Record course information\\
\hline
Attribute & courseId&\\
\hline
Description & The unique identifier of a course&\\
\hline
Type & \texttt{VARCHAR(16)}&\\
\hline
Allowed Values &N/A &\\
\hline
Default Value &N/A &\\
\hline
Constraints & \texttt{NOT NULL}&\\
\hline
Source & Uniquely generated when a course outline is uploaded&\\
\hline
Usage & Record course information &\\
\hline
\end{longtable}


\section{The Scope of the Product}
\subsection{Product Boundary}
\lips
\subsection{Product Use Case Table}
\lips
\subsection{Individual Product Use Cases (PUC's)}
\lips

\section{Functional Requirements}
\subsection{Authentication}
\begin{itemize}[leftmargin=16.5mm,labelsep=4mm,label=\rthereqnum]

\item
The user could create an account with a username and a password.

\textbf{Rationale:} Users would need an account protected by a password to keep their schedule personal and private.

\textbf{Priority:} HIGH
\item
The user could log in to their account by providing a corresponding username and password.

\textbf{Rationale:} Users would need to access their accounts.

\textbf{Priority:} HIGH
\item
The user should be able to log out of the account.

\textbf{Rationale:} Users do not have to stay logged in.

\textbf{Priority:} HIGH
\item
The user should have access to their scheduling information once logged in.

\textbf{Rationale:} User's scheduling information should be available to them.

\textbf{Priority:} HIGH

\end{itemize}


\subsection{User input}
\begin{itemize}[leftmargin=16.5mm,labelsep=4mm,label=\rthereqnum]

\item[\rthereqnum \label{R_upload_pdf}]
The user could upload multiple PDF files containing course outlines to the system.

\textbf{Rationale:} Multiple courses would have multiple course outlines.

\textbf{Priority:} HIGH
\item
The system prompts users to choose their preferred study interval at which the system tends to allocate study sessions.

\textbf{Rationale:} The user would want to have their study session schedules at their preferred time.

\textbf{Priority:} MEDIUM
\item
The user could change their preferred study interval.

\textbf{Rationale:} The user's preferred study interval could change over time. 

\textbf{Priority:} MEDIUM
\item
The system prompts users to set their preferred Pomodoro intervals. 

\textbf{Rationale:} The user might have preferred Pomodoro intervals.

\textbf{Priority:} MEDIUM
\item
The user could change their Pomodoro interval.

\textbf{Rationale:} The user's preferred Pomodoro intervals could change over time.

\textbf{Priority:} MEDIUM
\item
The system supports a friend list allowing users to send friend requests.

\textbf{Rationale:} The user needs to send friend requests to add friends.

\textbf{Priority:} LOW
\item
The user could accept an incoming friend request.

\textbf{Rationale:} The user could accept the friend request if they wish to befriend the request sender.

\textbf{Priority:} LOW
\item
The user could reject an incoming friend request.

\textbf{Rationale:} The user could reject the friend request if they wish not to befriend the request sender.

\textbf{Priority:} LOW
\item
The user could change the progress of each task

\textbf{Rationale:} The progress of each task should align with reality.

\textbf{Priority:} HIGH
\item
On finishing each sub-task, the system should ask for the user's feedback on whether the pace is comfortable.

\textbf{Rationale:} The schedule needs feedback to adjust the pace.

\textbf{Priority:} HIGH
\end{itemize}

\subsection{Data}
\begin{itemize}[leftmargin=16.5mm,labelsep=4mm,label=\rthereqnum]
\item
The system could extract information including \texttt{courseName, taskType, taskName, weight, deadline} from the uploaded course outline PDF files.

\textbf{Rationale:} Course outlines would be provided as PDF files containing course information.

\textbf{Priority:} HIGH
\item
The system could generate a list of tasks from uploaded PDFs.

\textbf{Rationale:} To generate a study plan, tasks to be finished are needed.

\textbf{Priority:} HIGH
\item[\rthereqnum \label{R_prioritize_task}]
The system could assign priority to each task generated.

\textbf{Rationale:} From priority the system could allocate adequate time to finish each task.

\textbf{Priority:} HIGH
\item
The system could detect the user's attention level.

\textbf{Rationale:} The user's attention level is needed to allocate rest time.

\textbf{Priority:} MEDIUM
\item
The system should notify the user when the user's attention level is low.

\textbf{Rationale:} The system would keep the user engaged in their study task.

\textbf{Priority:} MEDIUM
\item[\rthereqnum \label{R_check_task_progress}]
The user should be able to view the progress of each task.

\textbf{Rationale:} The user would want to see their progress.

\textbf{Priority:} HIGH
\item
With authentication, the system could import event and schedule data from other calendar apps including Calendar, Outlook, and Google Calendar.

\textbf{Rationale:} Study session allocation would consider avoiding conflicts with existing events.

\textbf{Priority:} MEDIUM
\item
With authentication, the system could export event and schedule data to other calendar apps including Calendar, Outlook, and Google Calendar.

\textbf{Rationale:} The user would want to use their own calendar of choice.

\textbf{Priority:} MEDIUM
\item
The user should be able to export their study plan as a PDF.

\textbf{Rationale:} The user could want to have an overview of their study plan offline.

\textbf{Priority:} LOW
\end{itemize}

\subsection{Scheduling}
\begin{itemize}[leftmargin=16.5mm,labelsep=4mm,label=\rthereqnum]
\item
The system could calculate the estimated time needed to finish each task. 

\textbf{Rationale:} Total required tie is needed to decide the amount of subtasks needed.

\textbf{Priority:} HIGH
\item
The user could regenerate a study plan accordingly once a task's progress is changed by the user.

\textbf{Rationale:} Previously generated study plans could be outdated.

\textbf{Priority:} HIGH
\item
The estimated time needed to finish each task would dynamically adjust based on user feedback on past task completion progress.

\textbf{Rationale:} Estimated time needed may not be perfect.

\textbf{Priority:} HIGH
\item
The system could prioritize tasks extracted based on \texttt{taskType, weight} and \texttt{deadline}.

\textbf{Rationale:} The system would ensure high-priority tasks are finished on time.

\textbf{Priority:} HIGH
\item
With data containing task information, preferred study time, schedule, and Pomodoro intervals gathered the system could generate a detailed study plan containing multiple sub-tasks for each task in the course outline available in-app and export it to other calendars.

\textbf{Rationale:} A detailed study plan contains all the sub-tasks and their settings.

\textbf{Priority:} HIGH
\item
The system would have a Pomodoro clock ready for each sub-task.

\textbf{Rationale:} A Pomodoro clock would help the user balance their study and rest.

\textbf{Priority:} MEDIUM
\item
With a detailed study plan, the system could pair friends from the friend list with similar study plans to have video-based online study sessions.

\textbf{Rationale:} Friends with similar tasks at similar times could collaborate and inspire each other.

\textbf{Priority:} LOW
\end{itemize}
\section{Look and Feel Requirements}
\subsection{Appearance Requirements}
\lips
\subsection{Style Requirements}
\lips

\section{Usability and Humanity Requirements}
\subsection{Ease of Use Requirements}
\begin{itemize}[leftmargin=16.5mm,labelsep=4mm,label=\rtheuhrnum]
\item The navigation must be intuitive to use by users with \texttt{MIN\_EDUCATION} education background.

\textbf{Fit Criterion:} \texttt{MIN\_UNDERSTAND\%} of users with at least \texttt{MIN\_EDUCATION} of education could navigate through functions within \texttt{MAX\_TRIAL\_TIME} of exploring.

\begin{itemize}
    \item \( U \): Total number of users
    \item \( E \): The set of users with at least \texttt{MIN\_EDUCATION}
    \item \( N \): The number of users who can navigate through functions within \texttt{MAX\_TRIAL\_TIME}
\end{itemize}
\[
    N \geq \left( \frac{\texttt{MIN\_UNDERSTAND\%}}{100} \right) \cdot |E|
\]
\end{itemize}
\subsection{Personalization and Internationalization Requirements}
N/A
\subsection{Learning Requirements}
\begin{itemize}[leftmargin=16.5mm,labelsep=4mm,label=\rtheuhrnum]
\item
The system must be understood by users within \texttt{MAX\_TRIAL\_TIME} of exploring.

\textbf{Fit Criterion:} \texttt{MIN\_UNDERSTAND\%} of users with at least \texttt{MIN\_EDUCATION} of education could understand the system within \texttt{MAX\_TRIAL\_TIME} of exploring.
\begin{itemize}
    \item \( U \): Total number of users
    \item \( E \): The set of users with at least \texttt{MIN\_EDUCATION}
    \item \( N \): The number of users who can navigate through functions within \texttt{MAX\_TRIAL\_TIME}
\end{itemize}

\[
    N \geq \left( \frac{\texttt{MIN\_UNDERSTAND\%}}{100} \right) \cdot |E|
\]
\end{itemize}
\subsection{Understandability and Politeness Requirements}
\begin{itemize}[leftmargin=16.5mm,labelsep=4mm,label=\rtheuhrnum]
\item
The language in the app must be grammatically correct \texttt{MIN\_grammar\%} of the time.

\textbf{Fit Criterion:} A group of \texttt{MIN\_TESTER\_NUM}  users could find at most \texttt{MAX\_BAD\_GRAMMAR} of grammar mistakes.

\item
The language in the app must be non-offensive.

\textbf{Fit Criterion:} A group of \texttt{MIN\_TESTER\_NUM} users could find no more than \texttt{MAX\_OFFENSIVE} of offensive messages.
\begin{itemize}
    \item \( U \): Total number of users testing the system
    \item \( O \): The number of offensive messages found by a group of users
    \item \( G \): The group of at least \texttt{MIN\_TESTER\_NUM} users
\end{itemize}
\[
    O \leq \texttt{MAX\_OFFENSIVE}
\]
\end{itemize}
\subsection{Accessibility Requirements}
\begin{itemize}[leftmargin=16.5mm,labelsep=4mm,label=\rtheuhrnum]
    \item Color combinations used in the interface must be distinguished by users with color blindness.\\
    
    \textbf{Fit Criterion:} A group of \texttt{MIN\_TESTER\_NUM} users could find at most \texttt{MAX\_COLOR\_AMBIGUOUS} of indistinguishable color combinations in the UI filtered with Color Oracle.
    \begin{itemize}
        \item \( C \): The number of indistinguishable color combinations found by a group of users, filtered with Color Oracle
        \item \( G \): The group of at least \texttt{MIN\_TESTER\_NUM} users
    \end{itemize}
    \[
        C \leq \texttt{MAX\_COLOR\_AMBIGUOUS}
    \]
\end{itemize}
\section{Performance Requirements}
\subsection{Speed and Latency Requirements}
\lips
\subsection{Safety-Critical Requirements}
\lips
\subsection{Precision or Accuracy Requirements}
\lips
\subsection{Robustness or Fault-Tolerance Requirements}
\lips
\subsection{Capacity Requirements}
\lips
\subsection{Scalability or Extensibility Requirements}
\lips
\subsection{Longevity Requirements}
\lips

\section{Operational and Environmental Requirements}
\subsection{Expected Physical Environment}
\begin{itemize}[leftmargin=16.5mm,labelsep=4mm,label=\rtheoernum]
\item
The system must be operable under the same physical environment that the desktop computer running it is operable.

\textbf{Fit Criterion:} When the machine is operable, the system is operable at least \texttt{MIN\_OPERABLE\%} of time.
\begin{itemize}
    \item \( T \): Total time the machine is operable
    \item \( S \): Total time the system is operable when the machine is operable
\end{itemize}
\[
    \frac{S}{T} \geq \frac{\texttt{MIN\_OPERABLE\%}}{100}
\]
\end{itemize}
\subsection{Requirements for Interfacing with Adjacent Systems}
\begin{itemize}[leftmargin=16.5mm,labelsep=4mm,label=\rtheoernum]
\item
The system could interface with calendar APIs when called.

\textbf{Fit Criterion:} At least \texttt{MIN\_API\_SUCCESS\%} of requests made to supported calendar APIs are successful.
\begin{itemize}
    \item \( R \): Total number of requests made to the supported calendar APIs
    \item \( S \): Number of successful requests made to the supported calendar APIs
\end{itemize}
\[
    \frac{S}{R} \geq \frac{\texttt{MIN\_API\_SUCCESS\%}}{100}
\]
\end{itemize}
\subsection{Productization Requirements}
N/A
\subsection{Release Requirements}
\begin{itemize}[leftmargin=16.5mm,labelsep=4mm,label=\rtheoernum]
\item
The released version must pass all known regression tests.

\textbf{Fit Criterion:} The released version must pass \texttt{MIN\_REGRESSION\_PASS\%} of regression tests. 
\begin{itemize}
    \item \( T \): Total number of regression tests conducted on the released version
    \item \( P \): Number of regression tests passed by the released version
\end{itemize}
\[
    \frac{P}{T} \geq \frac{\texttt{MIN\_REGRESSION\_PASS\%}}{100}
\]
\end{itemize}
\section{Maintainability and Support Requirements}
\subsection{Maintenance Requirements}
\lips
\subsection{Supportability Requirements}
\lips
\subsection{Adaptability Requirements}
\lips

\section{Security Requirements}
\subsection{Access Requirements}
\begin{itemize}[leftmargin=16.5mm,labelsep=4mm,label=\rthesrnum]
\item
The system is accessible only if the correct combo of username and password is provided.

\textbf{Fit Criterion:} An error message suggesting an incorrect password or username is displayed.
\item
Users could not access other users' data.

\textbf{Fit Criterion:} Only data linked to the currently logged-in account could be displayed.
\end{itemize}
\subsection{Integrity Requirements}
\begin{itemize}[leftmargin=16.5mm,labelsep=4mm,label=\rthesrnum]
\item
No Unauthorised entity could modify the database.

\textbf{Fit Criterion:} A group of \texttt{MIN\_TESTER\_NUM} unauthorised users could not modify data.
\end{itemize}
\subsection{Privacy Requirements}
\begin{itemize}[leftmargin=16.5mm,labelsep=4mm,label=\rthesrnum]
\item
The system will not release user information to a third party.

\textbf{Fit Criterion:} The system does not provide user information to a third party.
\end{itemize}
\subsection{Audit Requirements}
N/A
\subsection{Immunity Requirements}
N/A

\section{Cultural Requirements}
\subsection{Cultural Requirements}
\lips

\section{Compliance Requirements}
\subsection{Legal Requirements}
\begin{itemize}[leftmargin=16.5mm,labelsep=4mm,label=\rthecrnum]
\item
The system must comply with local laws and regulations.

\textbf{Fit Criterion:} The system does not violate local laws or regulations.
\end{itemize}
\subsection{Standards Compliance Requirements}
\begin{itemize}[leftmargin=16.5mm,labelsep=4mm,label=\rthecrnum]
\item
The code must comply with \textit{Flack8} coding standards.

\textbf{Fit Criterion:} Merged code passes Flake8 linter workflow check.
\end{itemize}
\section{Open Issues}
\lips

\section{Off-the-Shelf Solutions}
\subsection{Ready-Made Products}
\lips
\subsection{Reusable Components}
\lips
\subsection{Products That Can Be Copied}
\lips

\section{New Problems}
\subsection{Effects on the Current Environment}
\lips
\subsection{Effects on the Installed Systems}
\lips
\subsection{Potential User Problems}
\lips
\subsection{Limitations in the Anticipated Implementation Environment That May
Inhibit the New Product}
\lips
\subsection{Follow-Up Problems}
\lips

\section{Tasks}
\subsection{Project Planning}
\lips
\subsection{Planning of the Development Phases}
\lips

\section{Migration to the New Product}
\subsection{Requirements for Migration to the New Product}
\lips
\subsection{Data That Has to be Modified or Translated for the New System}
\lips

\section{Costs}
\lips
\section{User Documentation and Training}
\subsection{User Documentation Requirements}
\lips
\subsection{Training Requirements}
\lips

\section{Waiting Room}
\lips

\section{Ideas for Solution}
\lips

\newpage{}
\section*{Appendix --- Reflection}

The information in this section will be used to evaluate the team members on the
graduate attribute of Lifelong Learning.  Please answer the following questions:

\begin{enumerate}
  \item What knowledge and skills will the team collectively need to acquire to
  successfully complete this capstone project?  Examples of possible knowledge
  to acquire include domain specific knowledge from the domain of your
  application, or software engineering knowledge, mechatronics knowledge or
  computer science knowledge.  Skills may be related to technology, or writing,
  or presentation, or team management, etc.  You should look to identify at
  least one item for each team member.
  \item For each of the knowledge areas and skills identified in the previous
  question, what are at least two approaches to acquiring the knowledge or
  mastering the skill?  Of the identified approaches, which will each team
  member pursue, and why did they make this choice?
\end{enumerate}

\end{document}