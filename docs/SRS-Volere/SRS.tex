% THIS DOCUMENT IS FOLLOWS THE VOLERE TEMPLATE BY Suzanne Robertson and James Robertson
% ONLY THE SECTION HEADINGS ARE PROVIDED
%
% Initial draft from https://github.com/Dieblich/volere
%
% Risks are removed because they are covered by the Hazard Analysis
\documentclass[12pt]{article}

\usepackage{booktabs}
\usepackage{tabularx}
\usepackage{hyperref}
\hypersetup{
    bookmarks=true,         % show bookmarks bar?
      colorlinks=true,      % false: boxed links; true: colored links
    linkcolor=red,          % color of internal links (change box color with linkbordercolor)
    citecolor=green,        % color of links to bibliography
    filecolor=magenta,      % color of file links
    urlcolor=cyan           % color of external links
}

\newcommand{\lips}{\textit{Insert your content here.}}

%% Comments

\usepackage{color}

\newif\ifcomments\commentstrue %displays comments
%\newif\ifcomments\commentsfalse %so that comments do not display

\ifcomments
\newcommand{\authornote}[3]{\textcolor{#1}{[#3 ---#2]}}
\newcommand{\todo}[1]{\textcolor{red}{[TODO: #1]}}
\else
\newcommand{\authornote}[3]{}
\newcommand{\todo}[1]{}
\fi

\newcommand{\wss}[1]{\authornote{blue}{SS}{#1}} 
\newcommand{\plt}[1]{\authornote{magenta}{TPLT}{#1}} %For explanation of the template
\newcommand{\an}[1]{\authornote{cyan}{Author}{#1}}

%% Common Parts

\newcommand{\progname}{Course Buddy} % PUT YOUR PROGRAM NAME HERE
\newcommand{\authname}{Team \#5, Overwatch League
\\ Jingyao, Qin
\\ Qianni, Wang
\\ Qiang, Gao
\\ Chenwei, Song
\\ Shuting, Shi
\\ } % AUTHOR NAMES                  

\usepackage{hyperref}
    \hypersetup{colorlinks=true, linkcolor=blue, citecolor=blue, filecolor=blue,
                urlcolor=blue, unicode=false}
    \urlstyle{same}
                                

\begin{document}

\title{Software Requirements Specification for \progname: subtitle describing software} 
\author{\authname}
\date{\today}
	
\maketitle

~\newpage

\pagenumbering{roman}

\tableofcontents

~\newpage

\section*{Revision History}

\begin{tabularx}{\textwidth}{p{3cm}p{2cm}X}
\toprule {\textbf{Date}} & {\textbf{Version}} & {\textbf{Notes}}\\
\midrule
Date 1 & 1.0 & Notes\\
Date 2 & 1.1 & Notes\\
\bottomrule
\end{tabularx}

~\\

~\newpage
\section{Purpose of the Project}
\subsection{User Business}
\lips
The users would be students in high school to college institutes all over the world dealing with multiple courses.

\subsection{Goals of the Project}
We aim to develop a tool to help students integrate multiple course information with their personal schedules to generate a customized study plan. The plan should dynamically self-adjust according to the user's preference and study efficiency so that students can finish their deliverables before the deadline at their own pace with minimal pressure.
\section{Stakeholders}
\subsection{Client}
\lips
\subsection{Customer}
\lips
\subsection{Other Stakeholders}
\lips
\subsection{Hands-On Users of the Project}
\lips
\subsection{Personas}
\lips
\subsection{Priorities Assigned to Users}
\lips
\subsection{User Participation}
\lips
\subsection{Maintenance Users and Service Technicians}
\lips

\section{Mandated Constraints}
\subsection{Solution Constraints}
The development of a fully functional product is required to be finished by February 5 when the Revision 0 Demonstration is scheduled.
\subsection{Implementation Environment of the Current System}
The project implementation would be done through VS Code in the beginning and converted into Github Codespace once set up gets finished to ensure consistent compiling performance among group members.
\subsection{Partner or Collaborative Applications}
Our system would import event data from and export generated studen plan to popular calendar applications like Google Calendar and Outlook Calendar. 
\subsection{Off-the-Shelf Software}
\subsubsection{StudySchedule.org}
StudySchedule is a free scheduling software dedicated to generating customized daily schedules for students to study for MCAT. They would ask the user to set up an account, pick study material from their MCAT resource library, and take a questionnaire on time constraints and pace preference. Students could view their progress and make adjustments as they wish.
\subsubsection{Taskade AI Genrator}
Taskade is a powerful team project management tool. The component Taskade AI is capable of generating tasks for given project topics, creating checklists, project plans, and calendar schedules. Taskade AI is also capable of summarising PDF files.
\subsection{Anticipated Workplace Environment}
Our web base app is anticipated to be compatible with main stream browsers including Chrome, Safari, Microsoft Edge, and Firefox on the latest version of Windows and macOS.
\subsection{Schedule Constraints}
Each member of our team would devote 8 hours per week to work on the project making a total of 40 hours per week.
\subsection{Budget Constraints}
The monetary expenditure for the entire project could not exceed \$750.
\subsection{Enterprise Constraints}
\lips

\section{Naming Conventions and Terminology}
\subsection{Glossary of All Terms, Including Acronyms, Used by Stakeholders
involved in the Project}
\lips

\section{Relevant Facts And Assumptions}
\subsection{Relevant Facts}
Manually reading through multiple course outlines, inputting all the deliverable information into the calendar, and crunch time has always been an inefficient part of academic life. Students wish to have a seamless tool integrating deadline management into their everyday lives without risking overdue penalties.
\subsection{Business Rules}
\lips
\subsection{Assumptions}
\subsubsection{}
Course outlines are available as PDF files.
\subsubsection{}
Users are using a mainstream Calendar application: Google Calendar, Outlook Calendar, Calendar.
\subsubsection{}
Users have a relatively stable weekly schedule.

\section{The Scope of the Work}
\subsection{The Current Situation}
\lips
\subsection{The Context of the Work}
\lips
\subsection{Work Partitioning}
\lips
\subsection{Specifying a Business Use Case (BUC)}
\lips

\section{Business Data Model and Data Dictionary}
\subsection{Business Data Model}
\lips
\subsection{Data Dictionary}
\lips

\section{The Scope of the Product}
\subsection{Product Boundary}
\lips
\subsection{Product Use Case Table}
\lips
\subsection{Individual Product Use Cases (PUC's)}
\lips

\section{Functional Requirements}
\subsection{Authentication}
\subsubsection{}
The user could create an account with a username and a password.
\subsubsection{}
The user could log in to their account by providing a corresponding username and password.
\subsubsection{}
The user could access their and only their scheduling information once logged in.


\subsection{User input}
\subsubsection{}
Without overwriting, the user could upload multiple \textit{.pdf} files containing course outlines.
\subsubsection{}
The system prompts users to choose their preferred study interval at which the system tends to allocate study sessions and allows changes later.
\subsubsection{}
The system prompts users to set their preferred Pomodoro intervals and allows changes later. 
\subsubsection{}
The system supports a friend list allowing users to send/reject/accept friend requests.
\subsubsection{}
On finishing each sub-task, the system would ask for the user's feedback on whether the pace i comfortable.

\subsection{Data}
\subsubsection{}
The system could extract information including \texttt{courseName, taskType, taskName, weight, deadline} from uploaded course outline \textit{.pdf} files.
\subsubsection{}
With authentication, the system could import event and schedule data from other calendar apps including Calendar, Outlook, and Google Calendar.

\subsubsection{}
With authentication, the system could export event and schedule data to other calendar apps including Calendar, Outlook, and Google Calendar.




\subsection{Scheduling}
\subsubsection{}
The system could calculate the estimated time needed to finish each task extracted from course outlines. 

\subsubsection{}
The estimated time needed to finish each task would dynamically adjust based on user feedback on past task completion progress.

\subsubsection{}
The system could prioritize tasks extracted based on \texttt{taskType, weight} and \texttt{deadline}.

\subsubsection{}
With data containing task information, preferred study time, schedule, and Pomodoro intervals gathered the system could generate a detailed study plan containing multiple sub-tasks for each task in the course outline available in-app and export it to other calendars.


\subsubsection{}
The system would have a Pomodoro clock ready for each sub-task.



\subsubsection{}
The system allows users to check their progress on each task and adjust the time needed.

\subsubsection{}
With a detailed study plan, the system could pair friends from the friend list with similar study plans to have video-based online study sessions.

\section{Look and Feel Requirements}
\subsection{Appearance Requirements}
\lips
\subsection{Style Requirements}
\lips

\section{Usability and Humanity Requirements}
\subsection{Ease of Use Requirements}
The system must be easy to use by users with at least a grade 9 education background.
\subsection{Personalization and Internationalization Requirements}
N/A
\subsection{Learning Requirements}
The system must be understood by users within 10 minutes of exploring.
\subsection{Understandability and Politeness Requirements}
\subsubsection{}
The language in the app must be grammatically correct 99\% of the time.
\subsubsection{}
The language in the app must be non-offensive.
\subsection{Accessibility Requirements}
Color combinations used in the interface must be distinguished by users with color blindness.

\section{Performance Requirements}
\subsection{Speed and Latency Requirements}
\lips
\subsection{Safety-Critical Requirements}
\lips
\subsection{Precision or Accuracy Requirements}
\lips
\subsection{Robustness or Fault-Tolerance Requirements}
\lips
\subsection{Capacity Requirements}
\lips
\subsection{Scalability or Extensibility Requirements}
\lips
\subsection{Longevity Requirements}
\lips

\section{Operational and Environmental Requirements}
\subsection{Expected Physical Environment}
The system must be operable under the same physical environment that the desktop computer running it is operable.

\subsection{Requirements for Interfacing with Adjacent Systems}
The system could interface with calendar APIs when called.
\subsection{Productization Requirements}
N/A
\subsection{Release Requirements}
The released version must pass all known regression tests.

\section{Maintainability and Support Requirements}
\subsection{Maintenance Requirements}
\lips
\subsection{Supportability Requirements}
\lips
\subsection{Adaptability Requirements}
\lips

\section{Security Requirements}
\subsection{Access Requirements}
\subsubsection{}
The system is accessible only if the correct combo of username and password is provided.
\subsubsection{}
Users could not access other users' data.
\subsection{Integrity Requirements}
No Unauthorised entity could modify the database.
\subsection{Privacy Requirements}
\subsubsection{}
The system will not release user information to a third party.
\subsubsection{}
user authentication information will be stored with encryption.
\subsection{Audit Requirements}
N/A
\subsection{Immunity Requirements}
N/A

\section{Cultural Requirements}
\subsection{Cultural Requirements}
\lips

\section{Compliance Requirements}
\subsection{Legal Requirements}
The system must comply with local laws and regulations.
\subsection{Standards Compliance Requirements}
The code must comply with \textit{Flack8} coding standards.

\section{Open Issues}
\lips

\section{Off-the-Shelf Solutions}
\subsection{Ready-Made Products}
\lips
\subsection{Reusable Components}
\lips
\subsection{Products That Can Be Copied}
\lips

\section{New Problems}
\subsection{Effects on the Current Environment}
\lips
\subsection{Effects on the Installed Systems}
\lips
\subsection{Potential User Problems}
\lips
\subsection{Limitations in the Anticipated Implementation Environment That May
Inhibit the New Product}
\lips
\subsection{Follow-Up Problems}
\lips

\section{Tasks}
\subsection{Project Planning}
\lips
\subsection{Planning of the Development Phases}
\lips

\section{Migration to the New Product}
\subsection{Requirements for Migration to the New Product}
\lips
\subsection{Data That Has to be Modified or Translated for the New System}
\lips

\section{Costs}
\lips
\section{User Documentation and Training}
\subsection{User Documentation Requirements}
\lips
\subsection{Training Requirements}
\lips

\section{Waiting Room}
\lips

\section{Ideas for Solution}
\lips

\newpage{}
\section*{Appendix --- Reflection}

The information in this section will be used to evaluate the team members on the
graduate attribute of Lifelong Learning.  Please answer the following questions:

\begin{enumerate}
  \item What knowledge and skills will the team collectively need to acquire to
  successfully complete this capstone project?  Examples of possible knowledge
  to acquire include domain specific knowledge from the domain of your
  application, or software engineering knowledge, mechatronics knowledge or
  computer science knowledge.  Skills may be related to technology, or writing,
  or presentation, or team management, etc.  You should look to identify at
  least one item for each team member.
  \item For each of the knowledge areas and skills identified in the previous
  question, what are at least two approaches to acquiring the knowledge or
  mastering the skill?  Of the identified approaches, which will each team
  member pursue, and why did they make this choice?
\end{enumerate}

\end{document}