\documentclass{article}

\usepackage{booktabs}
\usepackage{tabularx}
\usepackage{hyperref}
\usepackage{longtable}
\usepackage{geometry}
\usepackage{pdflscape}

\hypersetup{
    colorlinks=true,       % false: boxed links; true: colored links
    linkcolor=red,          % color of internal links (change box color with linkbordercolor)
    citecolor=green,        % color of links to bibliography
    filecolor=magenta,      % color of file links
    urlcolor=cyan           % color of external links
}

\title{Hazard Analysis\\\progname}

\author{\authname}

\date{}

%% Comments

\usepackage{color}

\newif\ifcomments\commentstrue %displays comments
%\newif\ifcomments\commentsfalse %so that comments do not display

\ifcomments
\newcommand{\authornote}[3]{\textcolor{#1}{[#3 ---#2]}}
\newcommand{\todo}[1]{\textcolor{red}{[TODO: #1]}}
\else
\newcommand{\authornote}[3]{}
\newcommand{\todo}[1]{}
\fi

\newcommand{\wss}[1]{\authornote{blue}{SS}{#1}} 
\newcommand{\plt}[1]{\authornote{magenta}{TPLT}{#1}} %For explanation of the template
\newcommand{\an}[1]{\authornote{cyan}{Author}{#1}}

%% Common Parts

\newcommand{\progname}{Course Buddy} % PUT YOUR PROGRAM NAME HERE
\newcommand{\authname}{Team \#5, Overwatch League
\\ Jingyao, Qin
\\ Qianni, Wang
\\ Qiang, Gao
\\ Chenwei, Song
\\ Shuting, Shi
\\ } % AUTHOR NAMES                  

\usepackage{hyperref}
    \hypersetup{colorlinks=true, linkcolor=blue, citecolor=blue, filecolor=blue,
                urlcolor=blue, unicode=false}
    \urlstyle{same}
                                

\begin{document}

\maketitle
\thispagestyle{empty}

~\newpage

\pagenumbering{roman}

\begin{table}[hp]
\caption{Revision History} \label{TblRevisionHistory}
\begin{tabularx}{\textwidth}{llX}
\toprule
\textbf{Date} & \textbf{Developer(s)} & \textbf{Change}\\
\midrule
2023/10/19 & Chenwei Song, Qiang Gao  & Initial draft of the document\\
... & ... & ...\\
... & ... & ...\\
\bottomrule
\end{tabularx}
\end{table}

~\newpage

\tableofcontents

~\newpage

\pagenumbering{arabic}

\wss{You are free to modify this template.}

\section{Introduction}

\wss{You can include your definition of what a hazard is here.}

\section{Scope and Purpose of Hazard Analysis}

\section{System Boundaries and Components}

\section{Critical Assumptions}

\wss{These assumptions that are made about the software or system.  You should
minimize the number of assumptions that remove potential hazards.  For instance,
you could assume a part will never fail, but it is generally better to include
this potential failure mode.}

\clearpage % Start a new page
\begin{landscape}
\hspace*{-4cm}
\section{Failure Mode and Effect Analysis}


\begin{longtable}{|p{2cm}|p{3cm}|p{3cm}|p{3cm}|p{3cm}|p{3cm}|p{2cm}|}
\hline
\textbf{Design Function} & \textbf{Failure Modes} & \textbf{Effects of Failure} & \textbf{Causes of Failure} & \textbf{Detection} & \textbf{Recommended Action} & \textbf{SR} \\
\hline
User Registration & Username not accepted & Inability to access the tool & Username already exists & Authentication system would check username uniqueness & Notify the user to choose another username & SR1 SR2 SR4 SR6 \\
\hline
User Login & Login failure & Denied access & Password mismatch & Authentication system would check username and password match & Provide password recovery & SR1 SR2 SR5 SR4 SR6 SR7 \\
\hline
Task Generation & No tasks generated & Disorganized schedule & PDF extraction module not recognizing certain tasks & User feedback & Systematic bug fixes & FR16 PAR1 \\
\hline
Task Prioritization & Incorrect prioritization & Mismanaged time & Algorithm error & User feedback & Refine prioritization algorithm & SR3 FR17 PAR1 \\
\hline
Progress Visualization & Inaccurate visuals & Misunderstanding of progress & Progress data not updated with user feedback & Regression tests & Make sure visualization module uses updated data & SR3 FR20 OER3 \\
\hline
Connected Learning & Connection issues & Isolation in learning & Network instability & Connectivity checks & Run a connectivity check when the user attempts connected learning & SLR1 \\
\hline
Export to Other Calendars & Export failure & Disconnected schedules & Compatibility issues & User feedback & Enhance compatibility layers & SR5 SR3 AR1 \\
\hline
Estimate Task Duration & Incorrect estimates & Misallocated time for tasks & Algorithm inaccuracies & User feedback & Refine estimation algorithms & FR14 \\
\hline
\end{longtable}
\end{landscape}
\clearpage % End the landscape page and start a new page

\section{Safety and Security Requirements}
\begin{itemize}
\item \textbf{SR1: Data Encryption}
  \begin{itemize}
  \item \textbf{Description:} Ensure data encryption during data transfers to prevent unauthorized access.
  \item \textbf{Fit Criteria:} Data being transferred should be encrypted using industry-standard algorithms, with no plain-text data leaks detected.
  \item \textbf{Function to Fulfill:} Implement encryption protocols in the data transfer modules.
  \end{itemize}
\item \textbf{SR2: Encrypted Data Storage}
  \begin{itemize}
  \item \textbf{Description:} Store user data in a hashed or encrypted format to prevent direct access.
  \item \textbf{Fit Criteria:} No user data should be retrievable in plain text from the storage systems.
  \item \textbf{Function to Fulfill:} Use encryption/hashing mechanisms in the data storage systems.
  \end{itemize}
\item \textbf{SR3: Audit Log Maintenance}
  \begin{itemize}
  \item \textbf{Description:} Maintain an audit log of all activities within the application for traceability and accountability.
  \item \textbf{Fit Criteria:} All user and system activities should be logged with time stamps and relevant meta-data.
  \item \textbf{Function to Fulfill:} Integrate an activity logger within the application framework.
  \end{itemize}
\item \textbf{SR4: Role-based Access Control}
  \begin{itemize}
  \item \textbf{Description:} Have a strict role-based access control to prevent unauthorized data manipulation.
  \item \textbf{Fit Criteria:} Different user roles should have differing access levels, with no unauthorized data access incidents.
  \item \textbf{Function to Fulfill:} Implement role-based access control mechanisms in the user management module.
  \end{itemize}
\item \textbf{SR5: Security Patches and Updates}
  \begin{itemize}
  \item \textbf{Description:} Provide regular security patches and updates to the software to rectify known vulnerabilities.
  \item \textbf{Fit Criteria:} No known vulnerability should persist in the system for more than a month without a patch.
  \item \textbf{Function to Fulfill:}Establish a dedicated security updates team.
  \end{itemize}
\item \textbf{SR6: Attack Prevention}
  \begin{itemize}
  \item \textbf{Description:} The system should protect authentication data from brute force attacks.
  \item \textbf{Fit Criteria:} Restriction after a certain number of failed login attempts; option for the user to unlock account via email or phone.
  \item \textbf{Function to Fulfill:}Implement rate-limiting to prevent brute force attacks.
  \end{itemize}
\item \textbf{SR7: Password Recovery}
  \begin{itemize}
  \item \textbf{Description:} The system should provide a mechanism for users to retrieve their passwords in case they forget them.
  \item \textbf{Fit Criteria:} A user who has forgotten their password should be able to receive a password reset link via their registered email. This link should expire after a certain duration.
  \item \textbf{Function to Fulfill:}Implement a password recovery module that generates and sends a time-bound password reset link to the user's registered email.
  \end{itemize}
\end{itemize}

\section{Roadmap}

\wss{Which safety requirements will be implemented as part of the capstone timeline?
Which requirements will be implemented in the future?}

\end{document}