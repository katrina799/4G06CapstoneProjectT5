\documentclass{article}

\usepackage{booktabs}
\usepackage{tabularx}
\usepackage{hyperref}
\usepackage{enumitem}
\usepackage{longtable}
\usepackage{geometry}
\usepackage{pdflscape}
\usepackage{everypage}
\usepackage{lipsum}

\hypersetup{
    colorlinks=true,       
    linkcolor=blue,          
    citecolor=green         
}

\newcommand{\Lpagenumber}{\ifdim\textwidth=\linewidth\else\bgroup
  \dimendef\margin=0 %use \margin instead of \dimen0
  \ifodd\value{page}\margin=\oddsidemargin
  \else\margin=\evensidemargin
  \fi
  \raisebox{\dimexpr -\topmargin-\headheight-\headsep-0.5\linewidth}[0pt][0pt]{%
    \rlap{\hspace{\dimexpr \margin+\textheight+\footskip}%
    \llap{\rotatebox{90}{\thepage}}}}%
\egroup\fi}
\AddEverypageHook{\Lpagenumber}%

\newcounter{srnum} % Security Requirement number
\newcommand{\rthesrnum}{SR\refstepcounter{srnum}\thesrnum:}
\newcommand{\srref}[1]{SR\ref{#1}}


\title{Hazard Analysis\\\progname}

\author{\authname}

\date{}

%% Comments

\usepackage{color}

\newif\ifcomments\commentstrue %displays comments
%\newif\ifcomments\commentsfalse %so that comments do not display

\ifcomments
\newcommand{\authornote}[3]{\textcolor{#1}{[#3 ---#2]}}
\newcommand{\todo}[1]{\textcolor{red}{[TODO: #1]}}
\else
\newcommand{\authornote}[3]{}
\newcommand{\todo}[1]{}
\fi

\newcommand{\wss}[1]{\authornote{blue}{SS}{#1}} 
\newcommand{\plt}[1]{\authornote{magenta}{TPLT}{#1}} %For explanation of the template
\newcommand{\an}[1]{\authornote{cyan}{Author}{#1}}

%% Common Parts

\newcommand{\progname}{Course Buddy} % PUT YOUR PROGRAM NAME HERE
\newcommand{\authname}{Team \#5, Overwatch League
\\ Jingyao, Qin
\\ Qianni, Wang
\\ Qiang, Gao
\\ Chenwei, Song
\\ Shuting, Shi
\\ } % AUTHOR NAMES                  

\usepackage{hyperref}
    \hypersetup{colorlinks=true, linkcolor=blue, citecolor=blue, filecolor=blue,
                urlcolor=blue, unicode=false}
    \urlstyle{same}
                                

\begin{document}

\maketitle
\thispagestyle{empty}

~\newpage

\pagenumbering{roman}

\begin{table}[hp]
\caption{Revision History} \label{TblRevisionHistory}
\begin{tabularx}{\textwidth}{llX}
\toprule
\textbf{Date} & \textbf{Developer(s)} & \textbf{Change}\\
\midrule
2023/10/19 & Chenwei Song, Qiang Gao  & Initial draft of the document\\
... & ... & ...\\
... & ... & ...\\
\bottomrule
\end{tabularx}
\end{table}

~\newpage

\tableofcontents

~\newpage

\pagenumbering{arabic}


\section{Introduction}
This document presents the hazard analysis of the Course Buddy application. The Course Buddy App is a software designed to aid students in their academic endeavors by enabling efficient task management, syllabus uploading, task generation, and prioritization based on machine learning algorithms.
A hazard in the context of the Course Buddy App is any characteristic that, when combined with external circumstances, can lead to loss or compromise in the system. Hazards might pertain to data safety (protecting user data) and security (ensuring unauthorized access is prevented).

\section{Scope and Purpose of Hazard Analysis}
The primary aim of this document is to determine potential hazards within the system components, evaluate the effects and causes of failures, suggest mitigation measures, and determine resultant safety and security requirements.
\section{System Boundaries and Components}
\begin{enumerate}
    \item \textbf{The Course Buddy Application}: Installed on user devices, comprising both the user interface (front-end) and server interactions (back-end). The primary components are:
    \begin{itemize}
        \item Syllabus Uploading
        \item Task Generation
        \item Task Prioritization (ML-based)
        \item User Authentication \& Data Encryption
        \item Notification System
    \end{itemize}
    
    \item \textbf{The Physical Device (e.g., smartphone or tablet)}
    \item \textbf{The Database}: Where all academic data, syllabuses, and task information will be stored.
    \item \textbf{Backup Procedures}: Automated scripts for daily data backup.
\end{enumerate}

\section{Critical Assumptions}

\begin{enumerate}
    \item \textbf{Device Compatibility}: It is assumed that users will utilize devices that meet the application's minimum technical specifications.
    \item \textbf{Stable Internet Connectivity}: The app functions optimally with a stable internet connection, allowing for real-time data syncing, task updating, and notifications.
    \item \textbf{Database Reliability}: We assume that the third-party database provider consistently maintains industry-standard security measures and operational uptimes.
    \item \textbf{User Behavior}: It is assumed that users will not intentionally try to exploit or compromise the system. This includes attempting to bypass security protocols, introducing malicious software, or purposefully corrupting their data.
    \item \textbf{External Services}: Services and APIs the application relies upon (for tasks like ML processing or cloud operations) are assumed to be available and operational at all times.
    \item \textbf{Data Integrity}: It is assumed that the data being input by users, especially academic syllabuses or schedules, is accurate and up-to-date.
    \item \textbf{Hardware Durability}: We assume that user devices, such as smartphones or tablets, will not abruptly fail during application operations, which could lead to data loss or corruption.
\end{enumerate}

\clearpage % Start a new page

\section{Failure Mode and Effect Analysis}



\begin{longtable}{|p{1.5cm}|p{1.5cm}|p{1.5cm}|p{2cm}|p{2cm}|p{2cm}|p{1.5cm}|}
  \hline
  \textbf{Design Function} & \textbf{Failure Modes} & \textbf{Effects of Failure} & \textbf{Causes of Failure} & \textbf{Detection} & \textbf{Recommend\newline -ed Action} & \textbf{SR} \\
  \hline
  User Registration & Username not accepted & Inability to access the tool & Username already exists & Authenticati\-on system would check username\newline uniqueness & Notify the user to choose another username & \srref{Data_Encryption},\srref{Encrypted_Data_Storage}, \srref{Role-based_Access_Control},\srref{Attack_Prevention} \\
  \hline
  User Login & Login failure & Denied access & Password mismatch & Authenticat \-ion system would check username and password match & Provide password recovery & \srref{Data_Encryption},\srref{Encrypted_Data_Storage}, \srref{Security_Patches_and_Updates},\srref{Role-based_Access_Control}, \srref{Attack_Prevention},\srref{Password_Recovery} \\
  \hline
  Task Generation & No tasks generated & Disorgani -zed schedule & PDF extraction module not recognizing certain tasks & User feedback & Systematic bug fixes & \href{../SRS/SRS.pdf#page=28}{FR16}, \href{../SRS/SRS.pdf#page=34}{PAR1} \\
  \hline
  Task Prioritization & Incorrect prioritization & Mismana \-ged time & Algorithm error & User feedback & Refine prioritization algorithm & \srref{Audit_Log_Maintenance},\href{../SRS/SRS.pdf#page=28}{FR17}, \href{../SRS/SRS.pdf#page=34}{PAR1} \\
  \hline
  Progress Visualization & Inaccurate visuals & Misunders \-tanding of progress & Progress data not updated with user feedback & Regression tests & Make sure visualization module uses updated data & \srref{Audit_Log_Maintenance},\href{../SRS/SRS.pdf#page=28}{FR20}, \href{../SRS/SRS.pdf#page=36}{OER3}\\
  \hline
  Connected Learning & Connection issues & Isolation in learning & Network instability & Connectivity checks & Run a connectivity check when the user attempts connected learning & \href{../SRS/SRS.pdf#page=33}{SLR1} \\
  \hline
  Export to Other Calendars & Export failure & Disconnect \-ed schedules & Compatibility issues & User feedback & Enhance compatibility layers & \srref{Security_Patches_and_Updates},\srref{Audit_Log_Maintenance}, \href{../SRS/SRS.pdf#page=30}{AR1} \\
  \hline
  Estimate Task Duration & Incorrect estimates & Misalloca \-ted time for tasks & Algorithm inaccuracies & User feedback & Refine estimation algorithms & \href{../SRS/SRS.pdf#page=27}{FR14} \\
  \hline
  \end{longtable}
  
  \pagestyle{plain}%
  \clearpage % End the landscape page and start a new page

\section{Safety and Security Requirements}
\begin{itemize}
\item [\rthesrnum \label{Data_Encryption}] \textbf{Data Encryption}
  \begin{itemize}
  \item \textbf{Description:} Ensure data encryption during data transfers to prevent unauthorized access.
  \item \textbf{Fit Criteria:} Data being transferred should be encrypted using industry-standard algorithms, with no plain-text data leaks detected.
  \item \textbf{Function to Fulfill:} Implement encryption protocols in the data transfer modules.
  \end{itemize}
\item [\rthesrnum \label{Encrypted_Data_Storage}] \textbf{Encrypted Data Storage}
  \begin{itemize}
  \item \textbf{Description:} Store user data in a hashed or encrypted format to prevent direct access.
  \item \textbf{Fit Criteria:} No user data should be retrievable in plain text from the storage systems.
  \item \textbf{Function to Fulfill:} Use encryption/hashing mechanisms in the data storage systems.
  \end{itemize}
\item [\rthesrnum \label{Audit_Log_Maintenance}] \textbf{Audit Log Maintenance}
  \begin{itemize}
  \item \textbf{Description:} Maintain an audit log of all activities within the application for traceability and accountability.
  \item \textbf{Fit Criteria:} All user and system activities should be logged with time stamps and relevant meta-data.
  \item \textbf{Function to Fulfill:} Integrate an activity logger within the application framework.
  \end{itemize}
\item [\rthesrnum \label{Role-based_Access_Control}] \textbf{Role-based Access Control}
  \begin{itemize}
  \item \textbf{Description:} Have a strict role-based access control to prevent unauthorized data manipulation.
  \item \textbf{Fit Criteria:} Different user roles should have differing access levels, with no unauthorized data access incidents.
  \item \textbf{Function to Fulfill:} Implement role-based access control mechanisms in the user management module.
  \end{itemize}
\item [\rthesrnum \label{Security_Patches_and_Updates}] \textbf{Security Patches and Updates}
  \begin{itemize}
  \item \textbf{Description:} Provide regular security patches and updates to the software to rectify known vulnerabilities.
  \item \textbf{Fit Criteria:} No known vulnerability should persist in the system for more than a month without a patch.
  \item \textbf{Function to Fulfill:}Establish a dedicated security updates team.
  \end{itemize}
\item [\rthesrnum \label{Attack_Prevention}] \textbf{Attack Prevention}
  \begin{itemize}
  \item \textbf{Description:} The system should protect authentication data from brute force attacks.
  \item \textbf{Fit Criteria:} Restriction after a certain number of failed login attempts; option for the user to unlock account via email or phone.
  \item \textbf{Function to Fulfill:}Implement rate-limiting to prevent brute force attacks.
  \end{itemize}
\item [\rthesrnum \label{Password_Recovery}] \textbf{Password Recovery}
  \begin{itemize}
  \item \textbf{Description:} The system should provide a mechanism for users to retrieve their passwords in case they forget them.
  \item \textbf{Fit Criteria:} A user who has forgotten their password should be able to receive a password reset link via their registered email. This link should expire after a certain duration.
  \item \textbf{Function to Fulfill:}Implement a password recovery module that generates and sends a time-bound password reset link to the user's registered email.
  \end{itemize}
\end{itemize}

\section{Roadmap}

\subsection{Requirements to be Implemented as Part of the Capstone Timeline}
\begin{itemize}
    \item \srref{Data_Encryption} \\
    Ensure data encryption during data transfers to prevent unauthorized access. \\
    Integration of encryption protocols in the data transfer modules. \\
    
    \item \srref{Encrypted_Data_Storage} \\
    Store user data in a hashed or encrypted format. \\
    Utilize encryption/hashing mechanisms in the data storage systems. \\
    
    \item \srref{Audit_Log_Maintenance} \\
    Maintain an audit log of all application activities. \\
    Incorporate an activity logger within the application framework. \\
    
    \item \srref{Role-based_Access_Control} \\
    Implement strict role-based access control for data protection. \\
    Establish role-based access control mechanisms in the user management module. \\
    
    \item \srref{Attack_Prevention} \\
    Ensure protection of authentication data from brute force attacks. \\
    Introduce rate-limiting to thwart brute force attacks. \\
    
    \item \srref{Password_Recovery} \\
    Offer a mechanism for password retrieval. \\
    Develop a password recovery module to generate and send a time-bound password reset link. \\
\end{itemize}

\subsection{Requirements to be Implemented in the Future}
\begin{itemize}
    \item \srref{Security_Patches_and_Updates} \\
    Regularly roll out security patches and updates to fix known vulnerabilities. \\
    Form a dedicated security updates team to monitor, identify, and rectify vulnerabilities. \\
\end{itemize}

\end{document}