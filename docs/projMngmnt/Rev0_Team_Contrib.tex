\documentclass{article}

\usepackage{float}
\restylefloat{table}

\usepackage{booktabs}

\title{Team Contributions: Rev 0\\\progname}

\author{\authname}

\date{}

% \input{../Comments}
% %% Common Parts

\newcommand{\progname}{Course Buddy} % PUT YOUR PROGRAM NAME HERE
\newcommand{\authname}{Team \#5, Overwatch League
\\ Jingyao, Qin
\\ Qianni, Wang
\\ Qiang, Gao
\\ Chenwei, Song
\\ Shuting, Shi
\\ } % AUTHOR NAMES                  

\usepackage{hyperref}
    \hypersetup{colorlinks=true, linkcolor=blue, citecolor=blue, filecolor=blue,
                urlcolor=blue, unicode=false}
    \urlstyle{same}
                                


\begin{document}

\maketitle

\section{Demo Plans}

\wss{We will demo the new features of the application.  Pomodoro feature: show the user workflow for using Pomodoro to study. Course Page feature: demonstrate the mechanism of extract class information from the pdf user uploaded. To-do List feature: demonstrate the add/remove edit feature of the to-do list, drag and drop the different boxes on the kanban board. App integration feature: show user could redirect to other website/application pages by clicking the tiles }
\section{Meeting Attendance}

\wss{For each team member how many team meetings have they attended since the
POC demo.  This number should be determined from the meeting issues in the
team's repo.  The first entry in the table should be the total number of team
meetings held by the team.}

\begin{table}[H]
\centering
\begin{tabular}{ll}
\toprule
\textbf{Student} & \textbf{Meetings}\\
\midrule
Total & 5\\
Chenwei Song & 5\\
Qiang Gao & 5\\
Qianni Wang & 5\\
Shuting Shi & 5\\
Jingyao Qin & 5\\
\bottomrule
\end{tabular}
\end{table}

\wss{If needed, an explanation for the counts can be provided here.}

\section{Lecture Attendance}

\wss{For each team member how many lectures have they attended since the POC
demo.  This number should be determined from the lecture issues in the team's
repo.  The first entry in the table should be the total number of lectures since
the POC demo.}

\begin{table}[H]
\centering
\begin{tabular}{ll}
\toprule
\textbf{Student} & \textbf{Lectures}\\
\midrule
Total & 2\\
Chenwei Song & 2\\
Qiang Gao & 2\\
Qianni Wang & 2\\
Shuting Shi & 2\\
Jingyao Qin & 2\\
\bottomrule
\end{tabular}
\end{table}

\wss{If needed, an explanation for the lecture attendance can be provided here.}

\section{Commits}

\begin{table}[H]
\centering
\begin{tabular}{lll}
\toprule
\textbf{Student} & \textbf{Commits} & \textbf{Percent}\\
\midrule
Total & 361 & 100\% \\
Chenwei Song & 50 & 13.85\%\\
Qiang Gao & 63 & 17.45\%\\
Qianni Wang & 105 & 29.08\%\\
Shuting Shi & 68 & 18.83\%\\
Jingyao Qin & 75 & 20.77\%\\
\bottomrule
\end{tabular}
\end{table}

\section{Issue Tracker}


\begin{table}[H]
\centering
\begin{tabular}{lll}
\toprule
\textbf{Student} & \textbf{Authored (O+C)} & \textbf{Assigned (C only)}\\
\midrule
Chenwei Song & 16 & 34 \\
Qiang Gao & 5 & 37 \\
Qianni Wang & 52 & 62 \\
Shuting Shi & 14 & 43 \\
Jingyao Qin & 25 & 47 \\
\bottomrule
\end{tabular}
\end{table}


\section{CICD}

CI/CD is employed through various GitHub Actions workflows to automate code integration and deployment. The CI part includes lint checks on pull requests and pushes to the main branch, ensuring coding standards are met. If a pull request lacks labels, which are crucial for tracking and categorization, an automated comment prompts for labeling. Additionally, the CI verifies linked issues for completeness. Successful completions of these checks suggest that code changes are ready for integration or deployment, exemplifying how CI/CD facilitates maintaining code quality and streamlines the development cycle.

\end{document}