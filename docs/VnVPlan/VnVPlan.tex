\documentclass[12pt, titlepage]{article}

\usepackage{booktabs}
\usepackage{tabularx}
\usepackage{hyperref}
\usepackage{float}
\hypersetup{
    colorlinks,
    citecolor=blue,
    filecolor=black,
    linkcolor=red,
    urlcolor=blue
}
\usepackage[round]{natbib}

%% Comments

\usepackage{color}

\newif\ifcomments\commentstrue %displays comments
%\newif\ifcomments\commentsfalse %so that comments do not display

\ifcomments
\newcommand{\authornote}[3]{\textcolor{#1}{[#3 ---#2]}}
\newcommand{\todo}[1]{\textcolor{red}{[TODO: #1]}}
\else
\newcommand{\authornote}[3]{}
\newcommand{\todo}[1]{}
\fi

\newcommand{\wss}[1]{\authornote{blue}{SS}{#1}} 
\newcommand{\plt}[1]{\authornote{magenta}{TPLT}{#1}} %For explanation of the template
\newcommand{\an}[1]{\authornote{cyan}{Author}{#1}}

%% Common Parts

\newcommand{\progname}{Course Buddy} % PUT YOUR PROGRAM NAME HERE
\newcommand{\authname}{Team \#5, Overwatch League
\\ Jingyao, Qin
\\ Qianni, Wang
\\ Qiang, Gao
\\ Chenwei, Song
\\ Shuting, Shi
\\ } % AUTHOR NAMES                  

\usepackage{hyperref}
    \hypersetup{colorlinks=true, linkcolor=blue, citecolor=blue, filecolor=blue,
                urlcolor=blue, unicode=false}
    \urlstyle{same}
                                

\begin{document}

\title{Project Title: System Verification and Validation Plan for \progname{}} 
\author{\authname}
\date{\today}
	
\maketitle

\pagenumbering{roman}

\section*{Revision History}

\begin{tabularx}{\textwidth}{p{3cm}p{2cm}X}
\toprule {\bf Date} & {\bf Version} & {\bf Notes}\\
\midrule
Date 1 & 1.0 & Notes\\
Date 2 & 1.1 & Notes\\
\bottomrule
\end{tabularx}

~\\

\newpage

\tableofcontents

\listoftables


\listoffigures


\newpage

\section{Symbols, Abbreviations, and Acronyms}

\renewcommand{\arraystretch}{1.2}
\begin{tabular}{l l} 
  \toprule		
  \textbf{symbol} & \textbf{description}\\
  \midrule 
      UI & User Interface \\
      ML & Machine Learning\\
      API & Application Programming Interface\\
      HTTP & Hypertext Transfer Protocol\\
      PDF & Portable Document Format\\
      .csv & Comma-Separated Values \\
      .txt & Text file\\
  \bottomrule
\end{tabular}\\

\newpage

\pagenumbering{arabic}

This document outlines the Verification and Validation (V\&V) plan for the Course Buddy project developed by Team \#5, Overwatch League. The V\&V plan is a critical component of our project management and quality assurance processes, ensuring that Course Buddy not only meets its specified requirements but also fulfills the needs and expectations of its users and stakeholders.

\paragraph{Roadmap}
The V\&V plan is structured as follows:
\begin{enumerate}
    \item \textbf{Symbols, Abbreviations, and Acronyms}
    \item \textbf{General Information}
    \item \textbf{Plan} 
    \item \textbf{System Test Description} 
    \item \textbf{Unit Test Description}
\end{enumerate}
\section{General Information}

\subsection{Summary}



\subsection{Objectives}




\subsection{Relevant Documentation}



\citet{SRS}



\section{Plan}

This section outlines the comprehensive strategy for verifying and validating the Course Buddy software, ensuring it aligns with specified requirements and design standards. The plan spans from team roles in verification to the utilization of various testing and verification tools.

\subsection{Verification and Validation Team}
     \begin{table}[H]
     \centering
     \begin{tabular}{l p{10cm}}
          \toprule
          \textbf{Name} & \textbf{Role and Specific Duties}\\
          \midrule
          Jinyao Qin & \textbf{Lead Verifier}: Oversees the entire process, coordinates with other team members, and ensures all verification steps are followed diligently.\\
          
          Qianni Wang & \textbf{Implementation Specialist}: Reviews the codebase to ensure it aligns with the documented requirements, also verifies the code's functionality, performance, and security aspects.\\
          
          Qiang Gao & \textbf{Implementation Specialist}: same as Qianni Wang\\
          
          Chenwei Song & \textbf{Manual Test Engineer}: Responsible for manual test cases, ensuring that all tests run in different environments, and reporting the results in an understandable format for the team.\\
          
          Shuting Shi & \textbf{Test Automation Engineer}: Responsible for automating test cases, ensuring that all tests run in different environments, and reporting the results in an understandable format for the team.\\
          \bottomrule
        \end{tabular}
     \caption{Verification and Validation Team Members and Their Roles}
     \end{table}

\subsection{SRS Verification Plan}

For the verification of the Software Requirements Specification (SRS) document, the following approaches will be adopted:

\begin{enumerate}
    \item \textbf{Peer Review:} The SRS will be reviewed by team members and classmates to identify any inconsistencies, ambiguities, or missing requirements.
    \item \textbf{Expert Review:} Experts in software development will be consulted to ensure the requirements are complete and feasible.
    \item \textbf{Supervisor Review:} The SRS will be reviewed by our supervisor, who can provide valuable insights from a strategic and technical perspective.
    \item \textbf{Client Feedback:} The document will be shared with the client or stakeholders for their feedback, ensuring alignment with their expectations and needs.
    \item \textbf{Automated Analysis Tools:} Tools such as requirement management software will be used for tracing and managing requirements systematically.
\end{enumerate}

Additionally, an SRS checklist will be utilized to systematically verify the content of the SRS document:
Please see our \href{https://github.com/wangq131/4G06CapstoneProjectT5/blob/main/docs/SRS/SRS.pdf#page=2}{SRS.pdf} for more details.


\subsection{Design Verification Plan}

The design verification for our project will focus on ensuring that the design is user-friendly, intuitive, and aligns with the architectural requirements specified in the SRS. The verification plan will include the following key activities:

\begin{enumerate}
    \item \textbf{Peer Reviews:} The design documents and models will be reviewed by team members and classmates to critique and provide feedback on the design's usability, intuitiveness, and adherence to architectural requirements.
    \item \textbf{Supervisor Review:} The design will be presented to the project supervisor for a thorough review, focusing on adherence to technical specifications and project objectives.
    \item \textbf{Design Walkthroughs:} Scheduled sessions where the design team presents the design to the stakeholders, including peers and supervisors, for feedback and suggestions.
    \item \textbf{Prototype Testing:} Early versions of the design will be tested to gather quick feedback on the design's effectiveness and user experience.
    \item \textbf{Consistency Check:} Ensuring that the design remains consistent with the requirements and objectives outlined in the SRS document.
\end{enumerate}

To comprehensively verify the design, the following checklist will be used:

\begin{enumerate}
    \item \textbf{Design Documentation Review:}
    \begin{itemize}
        \item Check if the design documentation is complete and clearly describes the architecture, components, and interfaces.
        \item Ensure that the design aligns with the project's objectives and requirements specified in the SRS.
    \end{itemize}
    
    \item \textbf{User Interface (UI) and User Experience (UX) Evaluation:}
    \begin{itemize}
        \item Verify that the UI design is intuitive and user-friendly.
        \item Ensure UI consistency across different parts of the application.
        \item Assess the UX for compliance with common usability standards and practices.
    \end{itemize}

    \item \textbf{Architectural Conformity:}
    \begin{itemize}
        \item Confirm that the system architecture supports all the required functionalities.
        \item Check for scalability, maintainability, and flexibility of the design.
    \end{itemize}

    \item \textbf{Performance and Security Review:}
    \begin{itemize}
        \item Ensure that the design incorporates adequate performance optimizations.
        \item Review the design for potential security vulnerabilities and data protection measures.
    \end{itemize}

    \item \textbf{Compliance with Standards:}
    \begin{itemize}
        \item Verify adherence to relevant industry and design standards.
    \end{itemize}

    \item \textbf{Feedback Integration:}
    \begin{itemize}
        \item Check that feedback from previous reviews (by classmates, peers, or stakeholders) has been adequately incorporated into the design.
    \end{itemize}
\end{enumerate}


\subsection{Verification and Validation Plan Verification Plan}

The verification and validation (V\&V) plan for our project includes ensuring the integrity and effectiveness of the V\&V processes themselves. Given the importance of this plan in the overall project quality assurance, the following approaches will be employed:

\begin{enumerate}
    \item \textbf{Peer Review:} The V\&V plan will be reviewed by team members and classmates to identify any omissions or areas needing improvement.
    \item \textbf{Mutation Testing:} This technique will be applied to evaluate the ability of our test cases to detect faults deliberately injected into the code.
    \item \textbf{Iterative Feedback Incorporation:} Feedback from all review sessions and testing phases will be systematically incorporated to refine the V\&V plan.
\end{enumerate}

To systematically verify the V\&V plan, the following checklist will be used:

\begin{itemize}
    \item Is the plan comprehensive, covering all aspects of software verification and validation?
    \item Are the responsibilities and roles in the V\&V process clearly defined?
    \item Does the plan include a variety of testing methods (e.g., unit testing, integration testing, system testing)?
    \item Is there a clear process for incorporating feedback and continuous improvement in the V\&V process?
    \item Are there criteria defined for the success of each testing phase?
    \item Is mutation testing included to assess the thoroughness of the test cases?
    \item Are there measures in place to track and resolve any identified issues during the V\&V process?
    \item Does the plan align with the project's schedule, resources, and constraints?
\end{itemize}

\subsection{Implementation Verification Plan}

The Implementation Verification Plan will ensure that the software implementation adheres to the requirements and design specifications outlined in the SRS. Key components of this plan include:

\begin{itemize}
    \item \textbf{Unit Testing:} A comprehensive suite of unit tests, as detailed in the project's test plan, will validate individual components or modules of the software. /textit{PyTest}, a flexible and powerful testing tool, will be used for writing and executing these tests.
    \item \textbf{Static Analysis:} /textit{Pylint} and /textit{Flake8} will be employed for static code analysis to identify potential bugs, security vulnerabilities, and issues with code style and complexity.
    \item \textbf{Code Reviews and Walkthroughs:} Regularly scheduled code reviews and walkthroughs with team members and supervisors to inspect code quality, readability, and adherence to the Flask framework's best practices and design patterns.
    \item \textbf{Continuous Integration:} Automated build and testing processes will be implemented using tools like GitHub Actions, to ensure continuous code quality, integration, and deployment.
    \item \textbf{Performance Testing:} The use of tools like Locust for load testing will help evaluate the application's performance under various conditions, particularly focusing on how the Flask application handles concurrent requests and data processing.
\end{itemize}

\subsection{Automated Testing and Verification Tools}

For automated testing and verification in our Flask/Python project, the following tools will be employed:

\begin{itemize}
    \item \textbf{Unit Testing Framework:} /textit{PyTest} will be used for developing and running unit tests.
    \item \textbf{Profiling and Performance Tools:} Tools like /textit{cProfile} for Python will assist in identifying performance bottlenecks and optimizing code efficiency.
    \item \textbf{Static Code Analyzers:} /textit{Pylint} and /textit{Flake8} will be used to analyze Python code quality, adherence to coding standards, and identification of potential errors.
    \item \textbf{Continuous Integration:} GitHub Actions will automate the build, testing, and deployment process, ensuring continuous integration and delivery of the Python codebase.
    \item \textbf{Linters:} /textit{Flake8} will be used to enforce coding standards.
\end{itemize}


\subsection{Software Validation Plan}

The Software Validation Plan will focus on ensuring that the final product meets the requirements and expectations of the stakeholders. Key strategies include:

\begin{itemize}
    \item \textbf{Beta Testing:} Involvement of selected users in the beta testing phase to provide real-world feedback on the software's functionality and usability.
    \item \textbf{Stakeholder Review Sessions:} Regular review meetings with stakeholders to confirm that the software meets the intended requirements and use cases.
    \item \textbf{Demo to Supervisor:} A demonstration of the software to the project supervisor following the Rev 0 demo for feedback and validation.
    \item \textbf{Reference to SRS Verification:} Aligning the validation activities with the SRS verification efforts to ensure consistency in meeting the documented requirements.
\end{itemize}

\section{System Test Description}
	
\subsection{Tests for Functional Requirements}



\subsubsection{Area of Testing1}



\paragraph{Title for Test}

\begin{enumerate}

\item{test-id1\\}

Control: Manual versus Automatic
					
Initial State: 
					
Input: 
					
Output: 

Test Case Derivation: 
					
How test will be performed: 
					
\item{test-id2\\}

Control: Manual versus Automatic
					
Initial State: 
					
Input: 
					
Output: 

Test Case Derivation: 

How test will be performed: 

\end{enumerate}

\subsubsection{Area of Testing2}

...

\subsection{Tests for Nonfunctional Requirements}




\subsubsection{Area of Testing1}
		
\paragraph{Title for Test}

\begin{enumerate}

\item{test-id1\\}

Type: Functional, Dynamic, Manual, Static etc.
					
Initial State: 
					
Input/Condition: 
					
Output/Result: 
					
How test will be performed: 
					
\item{test-id2\\}

Type: Functional, Dynamic, Manual, Static etc.
					
Initial State: 
					
Input: 
					
Output: 
					
How test will be performed: 

\end{enumerate}

\subsubsection{Area of Testing2}

...

\subsection{Traceability Between Test Cases and Requirements}


\section{Unit Test Description}







\subsection{Unit Testing Scope}



\subsection{Tests for Functional Requirements}



\subsubsection{Module 1}


\begin{enumerate}

\item{test-id1\\}

Type: 

					
Initial State: 
					
Input: 
					
Output: 

Test Case Derivation: 

How test will be performed: 
					
\item{test-id2\\}

Type: 
					
Initial State: 
					
Input: 
					
Output: 

Test Case Derivation: 

How test will be performed: 

\item{...\\}
    
\end{enumerate}

\subsubsection{Module 2}

...

\subsection{Tests for Nonfunctional Requirements}





\subsubsection{Module ?}
		
\begin{enumerate}

\item{test-id1\\}

Type: 
					
Initial State: 
					
Input/Condition: 
					
Output/Result: 
					
How test will be performed: 
					
\item{test-id2\\}

Type: Functional, Dynamic, Manual, Static etc.
					
Initial State: 
					
Input: 
					
Output: 
					
How test will be performed: 

\end{enumerate}

\subsubsection{Module ?}

...

\subsection{Traceability Between Test Cases and Modules}


				
\bibliographystyle{plainnat}

\bibliography{../../refs/References}

\newpage

\section{Appendix}

This is where you can place additional information.

\subsection{Symbolic Parameters}

The definition of the test cases will call for SYMBOLIC\_CONSTANTS.
Their values are defined in this section for easy maintenance.

\subsection{Usability Survey Questions?}



\newpage{}
\section*{Appendix --- Reflection}

The information in this section will be used to evaluate the team members on the
graduate attribute of Lifelong Learning.  Please answer the following questions:

\newpage{}
\section*{Appendix --- Reflection}



The information in this section will be used to evaluate the team members on the
graduate attribute of Lifelong Learning.  Please answer the following questions:

\begin{enumerate}
  \item What knowledge and skills will the team collectively need to acquire to
  successfully complete the verification and validation of your project?
  Examples of possible knowledge and skills include dynamic testing knowledge,
  static testing knowledge, specific tool usage etc.  You should look to
  identify at least one item for each team member.
  \item For each of the knowledge areas and skills identified in the previous
  question, what are at least two approaches to acquiring the knowledge or
  mastering the skill?  Of the identified approaches, which will each team
  member pursue, and why did they make this choice?
\end{enumerate}

\end{document}