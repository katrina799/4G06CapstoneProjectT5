\documentclass[12pt, titlepage]{article}

\usepackage{booktabs}
\usepackage{tabularx}
\usepackage{hyperref}
\usepackage{float}
\hypersetup{
    colorlinks,
    citecolor=blue,
    filecolor=black,
    linkcolor=red,
    urlcolor=blue
}
\usepackage[round]{natbib}

%% Comments

\usepackage{color}

\newif\ifcomments\commentstrue %displays comments
%\newif\ifcomments\commentsfalse %so that comments do not display

\ifcomments
\newcommand{\authornote}[3]{\textcolor{#1}{[#3 ---#2]}}
\newcommand{\todo}[1]{\textcolor{red}{[TODO: #1]}}
\else
\newcommand{\authornote}[3]{}
\newcommand{\todo}[1]{}
\fi

\newcommand{\wss}[1]{\authornote{blue}{SS}{#1}} 
\newcommand{\plt}[1]{\authornote{magenta}{TPLT}{#1}} %For explanation of the template
\newcommand{\an}[1]{\authornote{cyan}{Author}{#1}}

%% Common Parts

\newcommand{\progname}{Course Buddy} % PUT YOUR PROGRAM NAME HERE
\newcommand{\authname}{Team \#5, Overwatch League
\\ Jingyao, Qin
\\ Qianni, Wang
\\ Qiang, Gao
\\ Chenwei, Song
\\ Shuting, Shi
\\ } % AUTHOR NAMES                  

\usepackage{hyperref}
    \hypersetup{colorlinks=true, linkcolor=blue, citecolor=blue, filecolor=blue,
                urlcolor=blue, unicode=false}
    \urlstyle{same}
                                

\begin{document}

\title{Project Title: System Verification and Validation Plan for \progname{}} 
\author{\authname}
\date{\today}
	
\maketitle

\pagenumbering{roman}

\section*{Revision History}

\begin{tabularx}{\textwidth}{p{3cm}p{2cm}X}
\toprule {\bf Date} & {\bf Version} & {\bf Notes}\\
\midrule
Date 1 & 1.0 & Notes\\
Date 2 & 1.1 & Notes\\
\bottomrule
\end{tabularx}

~\\

\newpage

\tableofcontents

\listoftables


\listoffigures


\newpage

\section{Symbols, Abbreviations, and Acronyms}

\renewcommand{\arraystretch}{1.2}
\begin{tabular}{l l} 
  \toprule		
  \textbf{symbol} & \textbf{description}\\
  \midrule 
      UI & User Interface \\
      ML & Machine Learning\\
      API & Application Programming Interface\\
      HTTP & Hypertext Transfer Protocol\\
  \bottomrule
\end{tabular}\\

\newpage

\pagenumbering{arabic}

This document outlines the Verification and Validation (V\&V) plan for the Course Buddy project developed by Team \#5, Overwatch League. The V\&V plan is a critical component of our project management and quality assurance processes, ensuring that Course Buddy not only meets its specified requirements but also fulfills the needs and expectations of its users and stakeholders.

\paragraph{Roadmap}
The V\&V plan is structured as follows:
\begin{enumerate}
    \item \textbf{Symbols, Abbreviations, and Acronyms}
    \item \textbf{General Information}
    \item \textbf{Plan} 
    \item \textbf{System Test Description} 
    \item \textbf{Unit Test Description}
\end{enumerate}
\section{General Information}

\subsection{Summary}

\wss{Say what software is being tested.  Give its name and a brief overview of
  its general functions.}

\subsection{Objectives}

\wss{State what is intended to be accomplished.  The objective will be around
  the qualities that are most important for your project.  You might have
  something like: ``build confidence in the software correctness,''
  ``demonstrate adequate usability.'' etc.  You won't list all of the qualities,
  just those that are most important.}

\wss{You should also list the objectives that are out of scope.  You don't have 
the resources to do everything, so what will you be leaving out.  For instance, 
if you are not going to verify the quality of usability, state this.  It is also 
worthwhile to justify why the objectives are left out.}

\wss{The objectives are important because they highlight that you are aware of 
limitations in your resources for verification and validation.  You can't do everything, 
so what are you going to prioritize?  As an example, if your system depends on an 
external library, you can explicitly state that you will assume that external library 
has already been verified by its implementation team.}

\subsection{Relevant Documentation}

\wss{Reference relevant documentation.  This will definitely include your SRS
  and your other project documents (design documents, like MG, MIS, etc).  You
  can include these even before they are written, since by the time the project
  is done, they will be written.}

\citet{SRS}

\wss{Don't just list the other documents.  You should explain why they are relevant and 
how they relate to your VnV efforts.}

\section{Plan}

This section outlines the comprehensive strategy for verifying and validating the Course Buddy software, ensuring its alignment with specified requirements and design standards. The plan spans from team roles in verification to the utilization of various testing and verification tools.

\subsection{Verification and Validation Team}
     \begin{table}[H]
     \centering
     \begin{tabular}{l p{10cm}}
          \toprule
          \textbf{Name} & \textbf{Role and Specific Duties}\\
          \midrule
          Jinyao Qin & \textbf{Lead Verifier}: Oversees the entire process, coordinates with other team members, and ensures all verification steps are followed diligently.\\
          
          Qianni Wang & \textbf{Implementation Specialist}: Reviews the codebase to ensure it aligns with the documented requirements, also verifies the code's functionality, performance, and security aspects.\\
          
          Qiang Gao & \textbf{Implementation Specialist}: same as Qianni Wang\\
          
          Chenwei Song & \textbf{Manual Test Engineer}: Responsible for manual test cases, ensuring that all tests run in different environments, and reporting the results in an understandable format for the team.\\
          
          Shuting Shi & \textbf{Test Automation Engineer}: Responsible for automating test cases, ensuring that all tests run in different environments, and reporting the results in an understandable format for the team.\\
          \bottomrule
        \end{tabular}
     \caption{Verification and Validation Team Members and Their Roles}
     \end{table}

\subsection{SRS Verification Plan}

For the verification of the Software Requirements Specification (SRS) document, the following approaches will be adopted:

\begin{enumerate}
    \item \textbf{Peer Review:} The SRS will be reviewed by team members and classmates to identify any inconsistencies, ambiguities, or missing requirements.
    \item \textbf{Expert Review:} Experts in software development will be consulted to ensure the requirements are complete and feasible.
    \item \textbf{Client Feedback:} The document will be shared with the client or stakeholders for their feedback, ensuring alignment with their expectations and needs.
    \item \textbf{Automated Analysis Tools:} Tools such as requirement management software will be used for tracing and managing requirements systematically.
\end{enumerate}

Additionally, an SRS checklist will be utilized to systematically verify the content of the SRS document:

\begin{itemize}
    \item \textbf{1. Purpose of the Project}
    \begin{itemize}
        \item 1.1. User Business
        \item 1.2. Goals of the Project
    \end{itemize}
    \item \textbf{2. Stakeholders}
    \begin{itemize}
        \item 2.1. Client
        \item 2.2. Customer
        \item 2.3. Other Stakeholders
        \item 2.4. Hands-On Users of the Project
        \item 2.5. User Participation
    \end{itemize}
    \item \textbf{6. The Scope of the Work}
    \begin{itemize}
        \item 6.1. The Current Situation
        \item 6.2. The Context of the Work
        \item 6.3. Specifying a Business Use Case (BUC)
    \end{itemize}
    \item \textbf{7. Business Data Model and Data Dictionary}
    \begin{itemize}
        \item 7.1. Business Data Model
        \item 7.2. Data Dictionary
    \end{itemize}
    \item \textbf{8. The Scope of the Product}
    \begin{itemize}
        \item 8.1. Product Boundary
        \item 8.2. Product Use Case Table
        \item 8.3. Individual Product Use Cases (PUC's)
    \end{itemize}
    \item \textbf{9. Functional Requirements}
    \begin{itemize}
        \item 9.1. Authentication
        \item 9.2. User Input
        \item 9.3. Data
        \item 9.4. Scheduling
    \end{itemize}
        \item \textbf{10. Look and Feel Requirements}
    \begin{itemize}
        \item 10.1. Appearance Requirements
        \item 10.2. Style Requirements
    \end{itemize}
    \item \textbf{11. Usability and Humanity Requirements}
    \begin{itemize}
        \item 11.1. Ease of Use Requirements
        \item 11.2. Personalization and Internationalization Requirements
        \item 11.3. Learning Requirements
        \item 11.4. Understandability and Politeness Requirements
        \item 11.5. Accessibility Requirements
    \end{itemize}
    \item \textbf{12. Performance Requirements}
    \begin{itemize}
        \item 12.1. Speed and Latency Requirements
        \item 12.2. Safety-Critical Requirements
        \item 12.3. Precision or Accuracy Requirements
        \item 12.4. Robustness or Fault-Tolerance Requirements
        \item 12.5. Capacity Requirements
        \item 12.6. Scalability or Extensibility Requirements
        \item 12.7. Longevity Requirements
    \end{itemize}
    \item \textbf{13. Operational and Environmental Requirements}
    \begin{itemize}
        \item 13.1. Expected Physical Environment
        \item 13.2. Requirements for Interfacing with Adjacent Systems
        \item 13.3. Productization Requirements
        \item 13.4. Release Requirements
    \end{itemize}
    \item \textbf{14. Maintainability and Support Requirements}
    \begin{itemize}
        \item 14.1. Maintenance Requirements
        \item 14.2. Supportability Requirements
        \item 14.3. Adaptability Requirements
    \end{itemize}
    \item \textbf{15. Security Requirements}
    \begin{itemize}
        \item 15.1. Access Requirements
        \item 15.2. Integrity Requirements
        \item 15.3. Privacy Requirements
        \item 15.4. Audit Requirements
        \item 15.5. Immunity Requirements
    \end{itemize}
    \item \textbf{16. Cultural Requirements}
    \item \textbf{17. Compliance Requirements}
    \begin{itemize}
        \item 17.1. Legal Requirements
        \item 17.2. Standards Compliance Requirements
    \end{itemize}
    \item \textbf{18. Open Issues}
    \item \textbf{19. Off-the-Shelf Solutions}
    \begin{itemize}
        \item 19.1. Ready-Made Products
        \item 19.2. Reusable Components
        \item 19.3. Products That Can Be Copied
    \end{itemize}
    \item \textbf{20. New Problems}
    \begin{itemize}
        \item 20.1. Effects on the Current Environment
        \item 20.2. Effects on the Installed Systems
        \item 20.3. Potential User Problems
        \item 20.4. Limitations in the Anticipated Implementation Environment
        \item 20.5. Follow-Up Problems
    \end{itemize}
    \item \textbf{21. Tasks}
    \begin{itemize}
        \item 21.1. Project Planning
        \item 21.2. Planning of the Development Phases
    \end{itemize}
    \item \textbf{22. Migration to the New Product}
    \item \textbf{23. Costs}
    \item \textbf{24. User Documentation and Training}
    \begin{itemize}
        \item 24.1. User Documentation Requirements
        \item 24.2. Training Requirements
    \end{itemize}
    \item \textbf{25. Waiting Room}
    \item \textbf{26. Ideas for Solutions}
\end{itemize}

\subsection{Design Verification Plan}

The design verification for our project will focus on ensuring that the design is user-friendly, intuitive, and aligns with the architectural requirements specified in the SRS. The verification plan will include the following key activities:

\begin{enumerate}
    \item \textbf{Peer Reviews:} The design documents and models will be reviewed by team members and classmates to critique and provide feedback on the design's usability, intuitiveness, and adherence to architectural requirements.
    \item \textbf{Design Walkthroughs:} Scheduled sessions where the design team presents the design to the stakeholders, including peers and supervisors, for feedback and suggestions.
    \item \textbf{Prototype Testing:} Early versions of the design will be tested to gather quick feedback on the design's effectiveness and user experience.
    \item \textbf{Consistency Check:} Ensuring that the design remains consistent with the requirements and objectives outlined in the SRS document.
\end{enumerate}

To comprehensively verify the design, the following checklist will be used:

\begin{enumerate}
    \item \textbf{Design Documentation Review:}
    \begin{itemize}
        \item Check if the design documentation is complete and clearly describes the architecture, components, and interfaces.
        \item Ensure that the design aligns with the project's objectives and requirements specified in the SRS.
    \end{itemize}
    
    \item \textbf{User Interface (UI) and User Experience (UX) Evaluation:}
    \begin{itemize}
        \item Verify that the UI design is intuitive and user-friendly.
        \item Ensure UI consistency across different parts of the application.
        \item Assess the UX for compliance with common usability standards and practices.
    \end{itemize}

    \item \textbf{Architectural Conformity:}
    \begin{itemize}
        \item Confirm that the system architecture supports all the required functionalities.
        \item Check for scalability, maintainability, and flexibility of the design.
    \end{itemize}

    \item \textbf{Performance and Security Review:}
    \begin{itemize}
        \item Ensure that the design incorporates adequate performance optimizations.
        \item Review the design for potential security vulnerabilities and data protection measures.
    \end{itemize}

    \item \textbf{Compliance with Standards:}
    \begin{itemize}
        \item Verify adherence to relevant industry and design standards.
    \end{itemize}

    \item \textbf{Feedback Integration:}
    \begin{itemize}
        \item Check that feedback from previous reviews (by classmates, peers, or stakeholders) has been adequately incorporated into the design.
    \end{itemize}
\end{enumerate}


\subsection{Verification and Validation Plan Verification Plan}

The verification and validation (V\&V) plan for our project includes ensuring the integrity and effectiveness of the V\&V processes themselves. Given the importance of this plan in the overall project quality assurance, the following approaches will be employed:

\begin{enumerate}
    \item \textbf{Peer Review:} The V\&V plan will be reviewed by team members and classmates to identify any omissions or areas needing improvement.
    \item \textbf{Mutation Testing:} This technique will be applied to evaluate the ability of our test cases to detect faults deliberately injected into the code.
    \item \textbf{Iterative Feedback Incorporation:} Feedback from all review sessions and testing phases will be systematically incorporated to refine the V\&V plan.
\end{enumerate}

To systematically verify the V\&V plan, the following checklist will be used:

\begin{itemize}
    \item Is the plan comprehensive, covering all aspects of software verification and validation?
    \item Are the responsibilities and roles in the V\&V process clearly defined?
    \item Does the plan include a variety of testing methods (e.g., unit testing, integration testing, system testing)?
    \item Is there a clear process for incorporating feedback and continuous improvement in the V\&V process?
    \item Are there criteria defined for the success of each testing phase?
    \item Is mutation testing included to assess the thoroughness of the test cases?
    \item Are there measures in place to track and resolve any identified issues during the V\&V process?
    \item Does the plan align with the project's schedule, resources, and constraints?
\end{itemize}

\subsection{Implementation Verification Plan}

The Implementation Verification Plan will ensure that the software implementation adheres to the requirements and design specifications outlined in the SRS. Key components of this plan include:

\begin{itemize}
    \item \textbf{Unit Testing:} A comprehensive suite of unit tests, as detailed in the project's test plan, will validate individual components or modules of the software. PyTest, a flexible and powerful testing tool, will be used for writing and executing these tests.
    \item \textbf{Static Analysis:} Pylint and Flake8 will be employed for static code analysis to identify potential bugs, security vulnerabilities, and issues with code style and complexity.
    \item \textbf{Code Reviews and Walkthroughs:} Regularly scheduled code reviews and walkthroughs with team members and supervisors to inspect code quality, readability, and adherence to the Flask framework's best practices and design patterns.
    \item \textbf{Continuous Integration:} Automated build and testing processes will be implemented using tools like GitHub Actions, to ensure continuous code quality, integration, and deployment.
    \item \textbf{Performance Testing:} The use of tools like Locust for load testing will help evaluate the application's performance under various conditions, particularly focusing on how the Flask application handles concurrent requests and data processing.
\end{itemize}

\subsection{Automated Testing and Verification Tools}

For automated testing and verification in our Flask/Python project, the following tools will be employed:

\begin{itemize}
    \item \textbf{Unit Testing Framework:} PyTest will be used for developing and running unit tests.
    \item \textbf{Profiling and Performance Tools:} Tools like cProfile for Python will assist in identifying performance bottlenecks and optimizing code efficiency.
    \item \textbf{Static Code Analyzers:} Pylint and Flake8 will be used to analyze Python code quality, adherence to coding standards, and identification of potential errors.
    \item \textbf{Continuous Integration:} GitHub Actions will automate the build, testing, and deployment process, ensuring continuous integration and delivery of the Python codebase.
    \item \textbf{Linters:} Flake8 will be used to enforce coding standards.
\end{itemize}


\subsection{Software Validation Plan}

The Software Validation Plan will focus on ensuring that the final product meets the requirements and expectations of the stakeholders. Key strategies include:

\begin{itemize}
    \item \textbf{Beta Testing:} Involvement of selected users in the beta testing phase to provide real-world feedback on the software's functionality and usability.
    \item \textbf{Stakeholder Review Sessions:} Regular review meetings with stakeholders to confirm that the software meets the intended requirements and use cases.
    \item \textbf{Demo to Supervisor:} A demonstration of the software to the project supervisor following the Rev 0 demo for feedback and validation.
    \item \textbf{Reference to SRS Verification:} Aligning the validation activities with the SRS verification efforts to ensure consistency in meeting the documented requirements.
\end{itemize}

\section{System Test Description}
	
\subsection{Tests for Functional Requirements}

\wss{Subsets of the tests may be in related, so this section is divided into
  different areas.  If there are no identifiable subsets for the tests, this
  level of document structure can be removed.}

\wss{Include a blurb here to explain why the subsections below
  cover the requirements.  References to the SRS would be good here.}

\subsubsection{Area of Testing1}

\wss{It would be nice to have a blurb here to explain why the subsections below
  cover the requirements.  References to the SRS would be good here.  If a section
  covers tests for input constraints, you should reference the data constraints
  table in the SRS.}
		
\paragraph{Title for Test}

\begin{enumerate}

\item{test-id1\\}

Control: Manual versus Automatic
					
Initial State: 
					
Input: 
					
Output: \wss{The expected result for the given inputs}

Test Case Derivation: \wss{Justify the expected value given in the Output field}
					
How test will be performed: 
					
\item{test-id2\\}

Control: Manual versus Automatic
					
Initial State: 
					
Input: 
					
Output: \wss{The expected result for the given inputs}

Test Case Derivation: \wss{Justify the expected value given in the Output field}

How test will be performed: 

\end{enumerate}

\subsubsection{Area of Testing2}

...

\subsection{Tests for Nonfunctional Requirements}

\wss{The nonfunctional requirements for accuracy will likely just reference the
  appropriate functional tests from above.  The test cases should mention
  reporting the relative error for these tests.  Not all projects will
  necessarily have nonfunctional requirements related to accuracy}

\wss{Tests related to usability could include conducting a usability test and
  survey.  The survey will be in the Appendix.}

\wss{Static tests, review, inspections, and walkthroughs, will not follow the
format for the tests given below.}

\subsubsection{Area of Testing1}
		
\paragraph{Title for Test}

\begin{enumerate}

\item{test-id1\\}

Type: Functional, Dynamic, Manual, Static etc.
					
Initial State: 
					
Input/Condition: 
					
Output/Result: 
					
How test will be performed: 
					
\item{test-id2\\}

Type: Functional, Dynamic, Manual, Static etc.
					
Initial State: 
					
Input: 
					
Output: 
					
How test will be performed: 

\end{enumerate}

\subsubsection{Area of Testing2}

...

\subsection{Traceability Between Test Cases and Requirements}

\wss{Provide a table that shows which test cases are supporting which
  requirements.}

\section{Unit Test Description}

\wss{This section should not be filled in until after the MIS (detailed design
  document) has been completed.}

\wss{Reference your MIS (detailed design document) and explain your overall
philosophy for test case selection.}  

\wss{To save space and time, it may be an option to provide less detail in this section.  
For the unit tests you can potentially layout your testing strategy here.  That is, you 
can explain how tests will be selected for each module.  For instance, your test building 
approach could be test cases for each access program, including one test for normal behaviour 
and as many tests as needed for edge cases.  Rather than create the details of the input 
and output here, you could point to the unit testing code.  For this to work, you code 
needs to be well-documented, with meaningful names for all of the tests.}

\subsection{Unit Testing Scope}

\wss{What modules are outside of the scope.  If there are modules that are
  developed by someone else, then you would say here if you aren't planning on
  verifying them.  There may also be modules that are part of your software, but
  have a lower priority for verification than others.  If this is the case,
  explain your rationale for the ranking of module importance.}

\subsection{Tests for Functional Requirements}

\wss{Most of the verification will be through automated unit testing.  If
  appropriate specific modules can be verified by a non-testing based
  technique.  That can also be documented in this section.}

\subsubsection{Module 1}

\wss{Include a blurb here to explain why the subsections below cover the module.
  References to the MIS would be good.  You will want tests from a black box
  perspective and from a white box perspective.  Explain to the reader how the
  tests were selected.}

\begin{enumerate}

\item{test-id1\\}

Type: \wss{Functional, Dynamic, Manual, Automatic, Static etc. Most will
  be automatic}
					
Initial State: 
					
Input: 
					
Output: \wss{The expected result for the given inputs}

Test Case Derivation: \wss{Justify the expected value given in the Output field}

How test will be performed: 
					
\item{test-id2\\}

Type: \wss{Functional, Dynamic, Manual, Automatic, Static etc. Most will
  be automatic}
					
Initial State: 
					
Input: 
					
Output: \wss{The expected result for the given inputs}

Test Case Derivation: \wss{Justify the expected value given in the Output field}

How test will be performed: 

\item{...\\}
    
\end{enumerate}

\subsubsection{Module 2}

...

\subsection{Tests for Nonfunctional Requirements}

\wss{If there is a module that needs to be independently assessed for
  performance, those test cases can go here.  In some projects, planning for
  nonfunctional tests of units will not be that relevant.}

\wss{These tests may involve collecting performance data from previously
  mentioned functional tests.}

\subsubsection{Module ?}
		
\begin{enumerate}

\item{test-id1\\}

Type: \wss{Functional, Dynamic, Manual, Automatic, Static etc. Most will
  be automatic}
					
Initial State: 
					
Input/Condition: 
					
Output/Result: 
					
How test will be performed: 
					
\item{test-id2\\}

Type: Functional, Dynamic, Manual, Static etc.
					
Initial State: 
					
Input: 
					
Output: 
					
How test will be performed: 

\end{enumerate}

\subsubsection{Module ?}

...

\subsection{Traceability Between Test Cases and Modules}

\wss{Provide evidence that all of the modules have been considered.}
				
\bibliographystyle{plainnat}

\bibliography{../../refs/References}

\newpage

\section{Appendix}

This is where you can place additional information.

\subsection{Symbolic Parameters}

The definition of the test cases will call for SYMBOLIC\_CONSTANTS.
Their values are defined in this section for easy maintenance.

\subsection{Usability Survey Questions?}

\wss{This is a section that would be appropriate for some projects.}

\newpage{}
\section*{Appendix --- Reflection}

The information in this section will be used to evaluate the team members on the
graduate attribute of Lifelong Learning.  Please answer the following questions:

\newpage{}
\section*{Appendix --- Reflection}

\wss{This section is not required for CAS 741}

The information in this section will be used to evaluate the team members on the
graduate attribute of Lifelong Learning.  Please answer the following questions:

\begin{enumerate}
  \item What knowledge and skills will the team collectively need to acquire to
  successfully complete the verification and validation of your project?
  Examples of possible knowledge and skills include dynamic testing knowledge,
  static testing knowledge, specific tool usage etc.  You should look to
  identify at least one item for each team member.
  \item For each of the knowledge areas and skills identified in the previous
  question, what are at least two approaches to acquiring the knowledge or
  mastering the skill?  Of the identified approaches, which will each team
  member pursue, and why did they make this choice?
\end{enumerate}

\end{document}