\documentclass{article}

\usepackage{booktabs}
\usepackage{tabularx}
\usepackage{array}
\usepackage{multirow}
\usepackage[margin=1.5in]{geometry}

\title{\textbf{Development Plan}\\\textbf{Project Name: Course Buddy}}


\author{\textbf{Team 5} \\ \\ Jingyao Qin, qinj15\\ Qianni Wang, wangq131\\ Shuting Shi, shis20\\ Chenwei Song, songc12\\  Qiang Gao, gaoq20}

\date{}

%% Comments

\usepackage{color}

\newif\ifcomments\commentstrue %displays comments
%\newif\ifcomments\commentsfalse %so that comments do not display

\ifcomments
\newcommand{\authornote}[3]{\textcolor{#1}{[#3 ---#2]}}
\newcommand{\todo}[1]{\textcolor{red}{[TODO: #1]}}
\else
\newcommand{\authornote}[3]{}
\newcommand{\todo}[1]{}
\fi

\newcommand{\wss}[1]{\authornote{blue}{SS}{#1}} 
\newcommand{\plt}[1]{\authornote{magenta}{TPLT}{#1}} %For explanation of the template
\newcommand{\an}[1]{\authornote{cyan}{Author}{#1}}

%% Common Parts

\newcommand{\progname}{Course Buddy} % PUT YOUR PROGRAM NAME HERE
\newcommand{\authname}{Team \#5, Overwatch League
\\ Jingyao, Qin
\\ Qianni, Wang
\\ Qiang, Gao
\\ Chenwei, Song
\\ Shuting, Shi
\\ } % AUTHOR NAMES                  

\usepackage{hyperref}
    \hypersetup{colorlinks=true, linkcolor=blue, citecolor=blue, filecolor=blue,
                urlcolor=blue, unicode=false}
    \urlstyle{same}
                                

\begin{document}

\maketitle

\begin{table}[hp]
\caption{Revision History} \label{TblRevisionHistory}
\begin{tabularx}{\textwidth}{llX}
\toprule
\textbf{Date} & \textbf{Developer(s)} & \textbf{Change}\\
\midrule
2023/9/24 & Jingyao Qin, Qianni Wang  & Description of changes\\
Date2 & Name(s) & Description of changes\\
... & ... & ...\\
\bottomrule
\end{tabularx}
\end{table}

The project is a time management tool designed to assist students in studying and completing coursework efficiently and in an organized manner.

\section{Team Meeting Plan}
Team meetings for Team 5 will occur at least three times a week, excluding SFWRENG 4G06 lectures and tutorials.Each meeting will be no longer than 50 minutes in length, with a preference for Microsoft Teams on-line meetings, and the rest of the meetings will take place off-line on the McMaster University campus, and will be held at the following times The meeting time will be chosen from the following time slots that have been agreed upon by the entire team: 
\begin{itemize}
    \item Tuesday 5:30 p.m. to 8:00 p.m.
    \item Wednesday 2:30 p.m. to 5:00 p.m.
    \item Thursday 7:00 p.m. to 8:00 p.m.
    \item Friday 7:00 p.m. to 8:00 p.m.
    \item Saturday 12:00 p.m. to 5:00 p.m.
    \item Sunday 12:00 p.m. to 5:00 p.m.
\end{itemize}
Meeting Types:
\begin{itemize}
    \item \textbf{Sprint Planning Meeting:} 
    This type of meeting is held every two weeks and is primarily used to discuss breaking down work tasks, assigning issue owners, setting issue deadlines, establishing short-term goals and plans, as well as defining the completion criteria and expectations for tasks. Team supervisor will attend this type of meeting.
    \item \textbf{Stand-up Meeting:}
   	This type of meeting takes place three times a week, on Tuesday, Thursday and Friday, typically lasting no more than half an hour. It is primarily used for each team member to report on their work for the week, provide updates on progress, outline their next steps, discuss whether they need assistance, and identify any obstacles.
    \item \textbf{General Meeting:}
    The frequency of this type of meeting is random, and sometimes not all team members need to participate. It may involve one-on-one discussions. The primary purpose of this meeting is to quickly address specific details or engage in in-depth discussions about a single issue. Attendees should coordinate whether it's necessary to communicate the key meeting content to the rest of the team.
\end{itemize}
\section{Team Communication Plan}
WeChat is a platform used by the team on a daily basis and this is the quickest way to connect with team members.Z WeChat is mainly used to discuss task assignments, meeting times, meeting locations and meeting durations, brainstorming about the project, and planning extracurricular team building activities. 

To better track project progress and work distributions, we use GitHub to manage issue assignments and deadlines, use GitHub issues to record team meeting notes, and also use GitHub Pull requests to review, modify, and approve PR requests that make changes to the project repository.

\section{Team Member Roles}

\begin{table}[ht]
  \centering
  \begin{tabular}{|p{3cm}|p{6cm}|p{3cm}|}
    \hline
    \textbf{Member Name} & \multicolumn{2}{c|}{\textbf{Roles}} \\
    \hline
    Jingyao Qin & Team Lead, Expert on LaTeX and Team Management & \multirow{5}{*}{\parbox{3cm}{Developer, Tester, Reviewer, PR Approver, Issue Creator, Team Meeting Host and Scribe.}}\\
    \cline{1-2} 
    Qianni Wang & Team Co-Lead, Expert on Git and Project Management & \\
    \cline{1-2} 
    Shuting Shi & Project Board Manager & \\
    \cline{1-2}
    Chenwei Song & Document Manager, Expert on Documentation & \\
    \cline{1-2} 
    Qiang Gao & Coordinator, Data Engineer & \\
    \hline
  \end{tabular}
  \caption{Team Roles}
\end{table}


\textbf{	Team Member Responsibilities:}
\begin{itemize}
	\item \textbf{All Team Members:} Responsible for coding, conducting testing, reviewing pull requests, and approving them only after verifying and testing the modified documents. Additionally, responsible for creating issues with the appropriate issue template and configuring project settings, assigning team members, and applying labels. Ensure the successful achievement of the project's goals by the end of the capstone course.
	\item \textbf{Roles Rotate within Team:}For every meeting, there will be a rotating host and scribe within the team, with the exception of sprint planning meetings.
    \item \textbf{Jingyao Qin:} responsible for liaising with supervisors, teaching assistants, and professors, scheduling team meetings, ensuring equitable distribution of work within the group, and tracking and ensuring that project milestones are within a reasonable process.
    \item \textbf{Qianni Wang:} Oversee GitHub repository management, ensure proper formatting and consistency of issues and pull requests, and maintain organization and structure within the project repository.
    \item \textbf{Shuting Shi:} responsible for hosting  Sprint Planning meeting ensuring that all created issues are correctly labelled, assigned, and associated with the project, and ensuring that each issue is in the right status column on the project board view.
    \item \textbf{Chenwei Song:} for ensuring that all documents in the main branch of the GitHub project repository are updated to the latest version and free of typos and grammar issues.
    \item \textbf{Qiang Gao:} Responsible for assisting the team lead in coordinating project tasks, tracking project progress, facilitating internal team communication to ensure smooth and timely information flow, managing team financial management.
    
\end{itemize}



\section{Work flow Plan}

\begin{itemize}
	\item How will you be using git, including branches, pull request, etc.?
	\item How will you be managing issues, including template issues, issue
	classification, etc.?
\end{itemize}

\section{Proof of Concept Demonstration Plan}

What is the main risk, or risks, for the success of your project?  What will you
demonstrate during your proof of concept demonstration to convince yourself that
you will be able to overcome this risk?


\section{Technology}

\begin{itemize}
\item Specific programming language
\item Specific linter tool (if appropriate)
\item Specific unit testing framework
\item Investigation of code coverage measuring tools
\item Specific plans for Continuous Integration (CI), or an explanation that CI
  is not being done
\item Specific performance measuring tools (like Valgrind), if
  appropriate
\item Libraries you will likely be using?
\item Tools you will likely be using?
\end{itemize}

\section{Coding Standard}

\section{Project Scheduling}

n this section, we encompassing significant milestones and associated deadlines. The project will be partitioned into 6 phases, each marked by a specific collection of actions and deliverables.

\subsection{Project Phases}

The project will be divided into the following phases:

\begin{enumerate}
  \item \textbf{Initiation Phase} (Duration: [September 5] - [September 25])
  \begin{itemize}
    \item Team building and project selection.
    \item Identify the project supervisor.
    \item Create a GitHub project repository.
  \end{itemize}

  \item \textbf{Planning Phase} (Duration: [September 26] - [October 11])
  \begin{itemize}
    \item Create a Development Plan.
    \item Define the problem and scope of the project.
    \item Identify stakeholders and investigate their needs.
    \item Create SRS Documentation.
  \end{itemize}

  \item \textbf{First Implementation Phase} (Duration: [October 12] - [November 24])
  \begin{itemize}
    \item Address the implementation challenges of the project.
    \item Perform Hazard Analysis.
    \item Create the Verification and Validation Plan.
    \item Present Proof of Concept Demonstration.
  \end{itemize}
  
  \item \textbf{Second Implementation Phase} (Duration: [November 25] - [January 17])
  \begin{itemize}
    \item Implement the main features of the project.
    \item Perform testing.
    \item Optimize the implementation details of the project.
    \item Create the Design Document (Version 0).
  \end{itemize}

  \item \textbf{Evaluation Phase} (Duration: [January 18] - [March 6])
  \begin{itemize}
    \item Evaluate the achievement of project goals.
    \item Perform risk and security assessments.
    \item Conduct user end testing assessments.
    \item Demonstrate Revision 0.
    \item Prepare the Verification and Validation Report.
  \end{itemize}

  \item \textbf{Closure Phase} (Duration: [March 7] - [April 4])
  \begin{itemize}
    \item Conduct the final demonstration.
    \item Get ready for the Expo Demonstration.
    \item Complete all documentation.
  \end{itemize}
\end{enumerate}


\subsection{Milestones and Deadlines}

The following are milestones and their respective deadlines for this project:

\begin{itemize}
  \item \textbf{Team Formed, Project Selected} - [September 18]
  \item \textbf{Problem Statement, POC Plan, Development Plan} - [October 4]
  \item \textbf{Requirement Document  Revision 0} - [October 11]
  \item \textbf{Hazard Analysis 0} - [October 20]
  \item \textbf{Verification and Validation Plan Revision 0} - [November 3]
  \item \textbf{Proof of Concept Demonstration} - [November 13-24]
  \item \textbf{Design Document Revision 0} - [January 17]
  \item \textbf{Revision 0 Demonstration} - [February 5-16]
  \item \textbf{Verification and Validation Report Revision 0} - [March 6]
  \item \textbf{Final Demonstration (Revision 1)} - [March 18-29]
  \item \textbf{EXPO Demonstration} - [April TBD]
  \item \textbf{Final Documentation (Revision 1)} - [April 4]
\end{itemize}

\end{document}